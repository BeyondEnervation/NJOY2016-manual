\section{UNRESR}
\label{sUNRESR}

\hypertarget{sUNRESRhy}{The}
UNRESR\index{UNRESR|textbf} module is used to produce effective self-shielded
\index{self-shielding} cross sections for resonance reactions in the
unresolved energy range.\index{unresolved resonance range} In ENDF-format
evaluations, the unresolved range begins at an energy where it is difficult
to measure individual resonances and extends to an energy where the effects
of fluctuations in the resonance cross sections become unimportant for
practical calculations.  As described in the ENDF format
manual,\cite{ENDF102} resonance information for this energy range is
given as average values for resonance widths and spacings together
with distribution functions for the widths and spacings.  This
representation can be converted into effective cross sections suitable
for codes that use the background cross section method, often
called the Bondarenko method,\cite{Bondarenko}\index{Bondarenko method}
using a method originally developed for the MC2
code\cite{MC2}\index{MC2} and extended for the ETOX
code\cite{ETOX}\index{ETOX}.  This unresolved-resonance method
has the following features:

\begin{itemize}
\begin{singlespace}
\item Flux-weighted cross sections are produced for the total,
    elastic, fission, and capture cross sections, including
    competition with inelastic scattering.

\item A current-weighted total cross section is produced for
    calculating the effective self-shielded transport cross section.

\item The energy grid used is consistent with the grid used by
   \hyperlink{sRECONRhy}{RECONR}.

\item The computed effective cross sections are written on the
   PENDF tape in a specially defined section (MF2, MT152) for
   use by other modules.

\item The accurate quadrature scheme from the MC2-2 code\cite{MC22}
   is used for computing averages over the ENDF statistical distribution
   functions.
\end{singlespace}
\end{itemize}

This chapter describes the UNRESR module in NJOY2016.0.

\subsection{Theory}
\label{ssUNRESR_theory}

In the unresolved energy range, it is not possible to define
precise values for the cross sections of the resonance reactions
$\sigma_x(E)$, where $x$ stands for the reaction type (total,
elastic, fission, or capture).  It is only possible to define
average values.  Of course, these average values should try
to preserve the reaction rate:

\begin{equation}
   \overline{\sigma}_{0x}(E^*)=\frac
     {\displaystyle\int_{E_1}^{E_2}\sigma_x(E)\,\phi_0(E)\,dE}
     {\displaystyle\int_{E_1}^{E_2}\phi_0(E)\,dE}\,\,,
\label{sbar0}
\end{equation}

\noindent
where $\phi_0(E)$ is the scalar flux, $E^*$ is an effective energy
in the range $[E_1,E_2]$, and the range $[E_1,E_2]$ is large
enough to hold many resonances but small with respect to
slowly varying functions of $E$.  In order to calculate
effective values for the transport cross section, it is
necessary to compute the current-weighted total cross
\index{current weighting}
section also.  It is given by

\begin{equation}
   \overline{\sigma}_{1t}(E^*)=\frac
     {\displaystyle\int_{E_1}^{E_2}\sigma_x(E)\,\phi_1(E)\,dE}
     {\displaystyle\int_{E_1}^{E_2}\phi_1(E)\,dE}\,\,,
\end{equation}

\noindent
where the P$_1$ component of the neutron flux, $\phi_1(E)$, is
proportional to the neutron current.  To proceed farther, it is
necessary to choose a model for the shape of $\phi_\ell(E)$ in the
vicinity of $E^*$.  The model used in UNRESR is based on the
B$_0$ approximation for large homogeneous systems and narrow
resonances:

\begin{equation}
   \phi_\ell(E)=\frac{C(E)}{[\,\Sigma_t(E)\,]^\ell}\,\,,
\end{equation}

\noindent
where $C(E)$ is a slowly varying function of $E$, and
$\Sigma_t(E)$ is the macroscopic total cross
section for the system.  In order to use this result in
Eq.~\ref{sbar0}, it is further assumed that the effects of other
isotopes in the mixture can be approximated by a constant called
$\sigma_0$ in the range $[E_1,E_2]$, or

\begin{equation}
   \phi_\ell(E)=\frac{C(E)}{[\,\sigma_0+\sigma_t(E)\,]^\ell}\,\,.
\label{flux}
\end{equation}

\noindent
Therefore, the effective cross sections in the unresolved range
are represented by

\begin{equation}
   \overline{\sigma}_{0x}(E^*)=\displaystyle\frac
     {\displaystyle\int_{E_1}^{E_2}
        \frac{\sigma_x(E)}{\sigma_0+\sigma_t(E)}C(E)\,dE}
     {\displaystyle\int_{E_1}^{E_2}
        \frac{1}{\sigma_0+\sigma_t(E)}C(E)\,dE}\,\,,
\end{equation}

\noindent
with $x$ being $t$ for total, $e$ for elastic, $f$ for fission,
and $\gamma$ for capture, and

\begin{equation}
   \overline{\sigma}_{1t}(E^*)=\displaystyle\frac
     {\displaystyle\int_{E_1}^{E_2}
        \frac{\sigma_x(E)}{[\,\sigma_0+\sigma_t(E)\,]^2}C(E)\,dE}
     {\displaystyle\int_{E_1}^{E_2}
        \frac{1}{[\,\sigma_0+\sigma_t(E)\,]^2}C(E)\,dE}\,\,.
\end{equation}

\noindent
This equation can also be written in the equivalent form

\begin{equation}
   \overline{\sigma}_{1t}(E^*)=\displaystyle\frac
     {\displaystyle\int_{E_1}^{E_2}
        \frac{1}{\sigma_0+\sigma_t(E)}C(E)\,dE}
     {\displaystyle\int_{E_1}^{E_2}
        \frac{1}{[\,\sigma_0+\sigma_t(E)\,]^2}C(E)\,dE}
     - \sigma_0\,\,.
\end{equation}

The parameter $\sigma_0$\index{$\sigma_0$} in Eq.~\ref{flux} deserves
more discussion.  It can be looked at as a parameter that controls
the depth of resonance dips in the flux.  When $\sigma_0$ is
large with respect to the peak cross sections of resonances in
$\sigma_t(E)$, the shape of the flux is essentially $C(E)$.
For smaller values of $\sigma_0$, dips will develop in the
flux that correspond to peaks in $\sigma_t$.  These dips will
cancel out part of the reaction rate in the region of the
peaks, thus leading to self-shielding\index{self-shielding}
of the cross section.  Analysis shows that it is possible to use
this single parameter to represent the effects of admixed materials
or the effects of neutron escape from an absorbing region.   See
the \hyperlink{sGROUPRhy}{GROUPR}\index{GROUPR} chapter
of this manual for additional details.

The cross sections that appear in the above integrals can be
written as the sum of a resonant part and a smooth part as
follows:

\begin{equation}
  \sigma_x(E)=b_x+\sigma_{Rx}(E)
    =b_x+\sum_s\sum_r\sigma_{xsr}(E{-}E_{sr})\,\,,
\end{equation}

\noindent
where $s$ is an index to a spin sequence, $r$ is an index to
a particular resonance in that spin sequence, and $E_{sr}$ is
the center energy for that resonance.  The smooth part $b_x$
can come from a smooth background given in the ENDF file,
and it also includes the potential scattering cross section
\index{potential scattering} $\sigma_p$ for the elastic and
total cross sections ($x{=}t$ and $x{=}e$).  In terms of the
smooth and resonant parts, the effective cross sections become

\begin{equation}
   \overline{\sigma}_{0x}(E^*)=b_x+\displaystyle\frac
     {\displaystyle\int_{E_1}^{E_2}
        \frac{\sigma_{Rx}(E)}{\overline{\sigma}+\sigma_{Rt}(E)}C(E)\,dE}
     {\displaystyle\int_{E_1}^{E_2}
        \frac{1}{\overline{\sigma}+\sigma_{Rt}(E)}C(E)\,dE}\,\,,
\label{R1}
\end{equation}

\noindent
and

\begin{equation}
   \overline{\sigma}_{1t}(E^*)=\displaystyle\frac
     {\displaystyle\int_{E_1}^{E_2}
        \frac{1}{\overline{\sigma}+\sigma_{Rt}(E)}C(E)\,dE}
     {\displaystyle\int_{E_1}^{E_2}
        \frac{1}{[\,\overline{\sigma}+\sigma_{Rt}(E)\,]^2}C(E)\,dE}
     - \sigma_0\,\,,
\label{R2}
\end{equation}

\noindent
where $\overline{\sigma}=b_t+\sigma_0$.  It is convenient to
transform the denominators of Eqs.~\ref{R1} and \ref{R2} into

\begin{equation}
   \int\frac{1}{\overline{\sigma}+\sigma_t}C\,dE
    =\frac{1}{\overline{\sigma}}\left\{
     \int C\,dE-\int\frac{\sigma_t}
        {\overline{\sigma}+\sigma_t}C\,dE\right\}\,\,,
\end{equation}

\noindent
and

\begin{equation}
   \int\frac{1}{[\overline{\sigma}+\sigma_t]^2}C\,dE
    =\frac{1}{\overline{\sigma}^2}\left\{
     \int C\,dE-\int\frac{\sigma_t}
            {\overline{\sigma}+\sigma_t}C\,dE
      -\int\frac{\overline{\sigma}\sigma_t}
        {[\overline{\sigma}+\sigma_t]^2}C\,dE\right\}\,\,.
\end{equation}

\noindent
Furthermore, since $C(E)$ is assumed to be a slowly-varying function
of $E$, it can be pulled out through all integrals and dropped.
The average cross sections become

\begin{equation}
  \overline{\sigma}_{0x}=b_x+\frac{\overline{\sigma}I_{0x}}
          {1-I_{0t}}\,\,,
\end{equation}

\noindent
and

\begin{equation}
   \overline{\sigma}_{1t}=b_t+\frac{\overline{\sigma}I_{1t}}
          {1-I_{0t}-I_{1t}} \,\,.
\end{equation}

\noindent
The last equation can also be written in the form

\begin{equation}
   \overline{\sigma}_{1t} = \overline{\sigma}\left[
     \frac{1-I_{0t}}{1-I_{0t}-I_{1t}}\right]-\sigma_0\,\,.
\end{equation}

\noindent
The average cross sections are thereby seen to depend on two types of
``fluctuation integrals:''\index{fluctuation integrals}

\begin{equation}
   I_{0x}=\frac{1}{E_2-E_1}
     \int_{E_1}^{E_2}\frac{\sigma_{Rx}(E)}
       {\overline{\sigma}+\sigma_{Rt}(E)}\,dE\,\,,
\end{equation}

\noindent
and

\begin{equation}
   I_{1t}=\frac{1}{E_2-E_1}
     \int_{E_1}^{E_2}\frac{\overline{\sigma}\sigma_{Rt}(E)}
       {[\,\overline{\sigma}+\sigma_{Rt}(E)\,]^2}\,dE\,\,,
\end{equation}

\noindent
where $x$ can take on the values $t$, $n$, $f$, or $\gamma$.  Note
that $I_{1t}{\leq}I_{0t}$, the difference increasing as $\sigma_0$
decreases from infinity.

Inserting the actual sums over resonances into the formula
for $I_{0x}$ gives

\begin{equation}
   I_{0x}=\frac{1}{E_2-E_1}
     \int_{E_1}^{E_2}\frac{\sum_{sr}\sigma_{xsr}(E-E_{sr})}
       {\overline{\sigma}+\sum_{sr}\sigma_{tsr}(E-E_{sr})}\,dE\,\,.
\end{equation}

\noindent
If the resonances were widely separated, only the ``self'' term
would be important, and one would obtain

\begin{equation}
   I_{0x}=\sum_{sr}\frac{1}{E_2-E_1}
     \int_{E_1}^{E_2}\frac{\sigma_{xsr}(E-E_{sr})}
       {\overline{\sigma}+\sigma_{tsr}(E-E_{sr})}\,dE\,\,.
\end{equation}

\noindent
Since the range of integration is large with respect to the width
of any one resonance, the variable of integration can be changed to
$\xi{=}E{-}E_{sr}$, and the limits on $\xi$ can be extended to
infinity.  For any one sequence, the interval $E_2{-}E_1$ is
equal to the average spacing of resonances in that sequence times
the number of resonances in the interval.  Therefore,

\begin{equation}
   I^I_{0x}=\sum_s\frac{1}{D_s}
    \frac{1}{N_s}\sum_r\int_{-\infty}^\infty
      \frac{\sigma_{xsr}(\xi)}
       {\overline{\sigma}+\sigma_{tsr}(\xi)} \,d\xi\,\,
\end{equation}

\noindent
where $D_s$ is the average spacing, and the ``I'' superscript
indicates that this is the ``isolated resonance'' result.
\index{isolated resonances}
Because there are assumed to be many resonances in the interval,
the sum over resonances can be changed to a multiple integration
over some characteristic set of parameters (such as widths) times
the probability of finding a resonance with some particular
values of the parameters:

\begin{equation}
   \frac{1}{N}\sum_{r\in s}f_r={<}f{>}_s=
    \int d\alpha P_s(\alpha)\int d\beta P_s(\beta)\cdots
         f(\alpha,\beta,\cdots)\,\,.
\end{equation}

\noindent
In the following text, this multiple integral (up to four fold)
will be abbreviated by writing the $\alpha$ integral only.
The final results for isolated resonances are as follows:

\begin{equation}
   I^I_{0x}=\sum_s\frac{1}{D_s}\int P(\alpha)
     \int_{-\infty}^\infty\frac{\sigma_{xs\alpha}(\xi)}
       {\overline{\sigma}+\sigma_{ts\alpha}(\xi)}\,d\xi\,d\alpha\,\,,
\end{equation}

\noindent
and

\begin{equation}
   I^I_{1t}=\sum_s\frac{1}{D_s}\int P(\alpha)
     \int_{-\infty}^\infty\frac{\overline{\sigma}\sigma_{ts\alpha}(\xi)}
       {[\,\overline{\sigma}+\sigma_{ts\alpha}(\xi)\,]^2}\,d\xi\,d\alpha\,\,.
\end{equation}

If the effects of overlap are too large to be neglected, overlap
\index{resonance overlap} corrections to the isolated resonance
result can be constructed using the continued-fraction generator

\begin{equation}
   \frac{1}{a+b}=\frac{1}{a}\left(1-\frac{b}{a+b}\right)\,\,.
\end{equation}

\noindent
Starting with the $I_0$ integrals,

\begin{eqnarray}
   \frac{\sum_{sr}\sigma_{xsr}}
     {\overline{\sigma}+\sum_{sr}\sigma_{tsr}}
   &=&\sum_{sr}
     \frac{\sigma_{xsr}}
      {\overline{\sigma}+\sigma_{tsr}}\Bigl\{\,1 \nonumber\\
   &-&\sum_{r'{\neq}r}\frac{\sigma_{tsr'}}
      {\overline{\sigma}+\sum\sigma_{tsr}}
      -\sum_{s'\neq s}\sum_{r'}\frac{\sigma_{ts'r'}}
         {\overline{\sigma}+\sum\sigma_{tsr}}\Bigr\}\,\,.
\end{eqnarray}

\noindent
Expand the second term in the braces to get

\begin{eqnarray}
   \frac{\sum_{sr}\sigma_{xsr}}
     {\overline{\sigma}+\sum_{sr}\sigma_{tsr}}
   &=&\sum_{sr}
     \frac{\sigma_{xsr}}
      {\overline{\sigma}+\sigma_{tsr}}\Bigl\{\,1 \nonumber\\
   &-&\sum_{r'{\neq}r}\frac{\sigma_{tsr'}}
      {\overline{\sigma}+\sigma_{tsr}+\sigma_{tsr'}} \nonumber\\
   & &\;\;\;\Bigl\{1
      -\sum_{{r''\neq r}\atop{r''\neq r'}}\frac{\sigma_{tsr''}}
         {\overline{\sigma}+\sum\sigma_{tsr}}
      -\sum_{s'\neq s}\sum_{r'}\frac{\sigma_{ts'r'}}
         {\overline{\sigma}+\sum\sigma_{tsr}}\Bigr\}\nonumber\\
   &-&\sum_{s'\neq s}\sum_{r'}\frac{\sigma_{ts'r'}}
         {\overline{\sigma}+\sum\sigma_{tsr}}\Bigr\}\,\,.
\end{eqnarray}

\noindent
Neglecting the products of three {\em different} resonances
in sequence $s$ gives

\begin{eqnarray}
   \frac{\sum_{sr}\sigma_{xsr}}
     {\overline{\sigma}+\sum_{sr}\sigma_{tsr}}
   &=&\sum_{sr}
     \frac{\sigma_{xsr}}
      {\overline{\sigma}+\sigma_{tsr}}  \nonumber\\
   &\times&\Bigl\{1-\sum_{r'{\neq}r}\frac{\sigma_{tsr'}}
      {\overline{\sigma}+\sigma_{tsr}+\sigma_{tsr'}}
                \Bigr\} \nonumber\\
   &\times& \Bigl[\, 1-\sum_{s'\neq s}\sum_{r'}\frac{\sigma_{ts'r'}}
         {\overline{\sigma}+\sum\sigma_{tsr}} \,\Bigr]\,\,.
\end{eqnarray}

\noindent
The factor before the opening brace is the isolated
resonance result, the factor in braces is the in-sequence
overlap correction, and the factor in brackets is the
sequence-sequence overlap correction.
\index{in-sequence overlap}
\index{sequence-sequence overlap}
Note that recursion can be used to refine the sequence-sequence correction
to any desired accuracy.  Similarly, the $I_1$ integral requires

\begin{eqnarray}
   \frac{\sum_{sr}\overline{\sigma}\sigma_{xsr}}
     {\left[\,\overline{\sigma}+\sum_{sr}\sigma_{tsr}\,\right]^2}
   &=&\sum_{sr}
     \frac{\overline{\sigma}\sigma_{xsr}}
      {[\,\overline{\sigma}+\sigma_{tsr}\,]^2} \Bigl[ \,1 \nonumber\\
   &-&\sum_{r'{\neq}r}\frac{\sigma_{tsr'}}
      {\overline{\sigma}+\sum\sigma_{tsr}}
      -\sum_{s'\neq s}\sum_{r'}\frac{\sigma_{ts'r'}}
         {\overline{\sigma}+\sum\sigma_{tsr}} \Bigr]^2\,\,.
\end{eqnarray}

\noindent
Once more, we expand the fraction and neglect terms that will result
in products of three or more different resonances in the
same sequence.  The result is

\begin{eqnarray}
   \frac{\sum_{sr}\overline{\sigma}\sigma_{xsr}}
     {\left[\,\overline{\sigma}+\sum_{sr}\sigma_{tsr}\,\right]^2}
   &=&\sum_{sr}
     \frac{\overline{\sigma}\sigma_{xsr}}
      {\left[\,\overline{\sigma}+\sigma_{tsr}\,\right]^2} \nonumber\\
   &\times&\Bigl\{1-2\sum_{r'{\neq}r}\frac{\sigma_{tsr'}}
      {\overline{\sigma}+\sigma_{tsr}+\sigma_{tsr'}}
     +\sum_{r'{\neq}r}\left(\frac{\sigma_{tsr'}}
      {\overline{\sigma}+\sigma_{tsr}+\sigma_{tsr'}}\right)^2\Bigr\}
       \nonumber\\
   &\times& \Bigl[\,1-\sum_{s'\neq s}\sum_{r'}\frac{\sigma_{ts'r'}}
         {\overline{\sigma}+\sum\sigma_{tsr}}\Bigr]\,\,,
\end{eqnarray}

\noindent
where in-sequence and sequence-sequence overlap
terms have been factored out.

The next step is to substitute these results back into the fluctuation
integrals $I_0$ and $I_1$.  The integrals over energy and the sums over
different resonances in each sequence can be handled as described
above for isolated resonances.  This procedure will result in three
different kinds of integrals.  The first kind includes the isolated
resonance integrals already considered above

\begin{eqnarray}
  B_{xs} &=& \frac{1}{E_2-E_1}\int_{E_1}^{E_2}\sum_r
    \frac{\sigma_{xsr}}{\overline{\sigma}+\sigma_{tsr}}\,dE \nonumber\\
  &=&\frac{1}{D_s}\int P(\alpha)\int_{-\infty}^\infty
    \frac{\sigma_{xs\alpha}(\xi)}{\overline{\sigma}+\sigma_{ts\alpha(\xi)}}
   \,d\xi\,d\alpha\,\,,
\end{eqnarray}

\noindent
and

\begin{eqnarray}
  D_{ts}&=&\frac{1}{E_2-E_1}\int_{E_1}^{E_2}\sum_r
    \frac{\overline{\sigma}\sigma_{tsr}}
       {[\,\overline{\sigma}+\sigma_{tsr}\,]^2}\,dE \nonumber\\
  &=&\frac{1}{D_s}\int P(\alpha)\int_{-\infty}^\infty
    \frac{\overline{\sigma}\sigma_{xs\alpha}(\xi)}
    {[\,\overline{\sigma}+\sigma_{ts\alpha}(\xi)\,]^2}
   \,d\xi\,d\alpha\,\,.
\end{eqnarray}

\noindent
Note that $D_t\le B_t$, the difference increasing as $\sigma_0$
decreases from infinity.

The next kind are the in-sequence overlap integrals.  The sum over $r'$ is
replaced by integrals over the probabilities of finding each partial width
and the probability of finding a resonance $r'$ at a distance $\eta$ from
resonance $r$.

\begin{eqnarray}
  V_{0xs} &=& \frac{1}{E_2-E_1}\int_{E_1}^{E_2}
     \sum_r\sum_{r'\neq r} \frac{\sigma_{xsr}}{\overline{\sigma}
       +\sigma_{tsr}}\frac{\sigma_{tsr'}}{\overline{\sigma}
       +\sigma_{tsr}+\sigma_{tsr'}}\,dE \nonumber\\
  &=& \frac{1}{D_s^2}\int P(\alpha)\int P(\beta)
         \int\int \Omega(\eta)\,\frac{\sigma_{xs\alpha}(\xi)}
           {\overline{\sigma}+\sigma_{ts\alpha}(\xi)} \nonumber\\
  & &\;\;\;\frac{\sigma_{ts\beta}(\xi-\eta)}
      {\overline{\sigma}+\sigma_{ts\alpha}(\xi)+\sigma_{ts\beta}(\xi-\eta)}
     \,d\eta\,d\xi\,d\beta\,d\alpha\,\,,
\end{eqnarray}

\noindent
where $\xi=E-E_{sr}$ and $\eta=E_{sr'}-E_{sr}$.  Similarly,

\begin{eqnarray}
  V_{1ts} &=& \frac{1}{E_2-E_1}\int_{E_1}^{E_2}\sum_r\sum_{r'\neq r}
    \frac{\overline{\sigma}\sigma_{tsr}}{[\,\overline{\sigma}
      +\sigma_{tsr}\,]^2}\Bigl\{2\frac{\sigma_{tsr'}}
       {\overline{\sigma}+\sigma_{tsr}+\sigma_{tsr'}} \nonumber\\
    & &\;\;\;-\Bigl(\frac{\sigma_{tsr'}}{\overline{\sigma}
        +\sigma_{tsr}+\sigma_{tsr'}}\Bigr)^2\Bigr\}\,dE \nonumber\\
  &=& \frac{1}{D_s^2}\int P(\alpha)\int P(\beta)\int\int
     \Omega(\eta)\,\frac{\overline{\sigma}\sigma_{ts\alpha}(\xi)}
      {[\,\overline{\sigma}+\sigma_{ts\alpha}(\xi)\,]^2} \nonumber\\
    & &\;\;\;\Bigl\{2\frac{\sigma_{ts\beta}(\xi-\eta)}
        {\overline{\sigma}+\sigma_{ts\alpha}(\xi)
          +\sigma_{ts\beta}(\xi-\eta)} \nonumber\\
  & &\;\;\;-\Bigl[\frac{\sigma_{ts\beta}(\xi-\eta)}
      {\overline{\sigma}+\sigma_{ts\alpha}(\xi)+\sigma_{ts\beta}(\xi-\eta)}
        \Bigr]^2 \Bigr\}\,d\eta\,d\xi\,d\beta\,d\alpha \,\,.
\end{eqnarray}

The final class of integrals includes the sequence-sequence overlap
corrections.  They are simplified by noting that the positions of
resonances in different spin sequences are uncorrelated.  Therefore,
$\Omega(\eta){=}1$, and the integral of the product reduces to the
product of the integrals.

Using the results and definitions from above, the fluctuation integrals become

\begin{equation}
  I_{0x} = \sum_s A_{xs}\,\,,
\label{Izero}
\end{equation}

\begin{equation}
  A_{xs} = (B_{xs}-V_{0xs})\Bigl[\,1-\sum_{s'\neq s}
     A_{ts'}\,\Bigr]\,\,,
\label{recursA}
\end{equation}

\noindent
and

\begin{equation}
  I_{1t} = \sum_s(D_{ts}-V_{1ts})\Bigl[\,1-\sum_{s'\neq s}
    A_{ts'}\Bigr]^2\,\,,
\label{Ione}
\end{equation}

\noindent
where Eq.~\ref{recursA} provides a recursive definition of the $A_{ts}$ for
the sequence-sequence corrections as well as the normal value of
$A_{xs}$.

These equations are formally exact for the sequence-sequence overlaps,
but in-sequence overlaps only include the interactions between
pairs of resonances.  Three different approximations to this
result are currently in use.


\paragraph{The MC2/ETOX Approximation}
The MC2\index{MC2} and ETOX\index{ETOX} codes use similar
approximations to the results above, except that MC2 does not
include a calculation of the current-weighted total cross section.
Both codes explicitly neglect the in-sequence overlap corrections.
This approximation was based on the assumption that resonance
repulsion would reduce the overlap between resonances in a
particular spin sequence, leaving the accidental close spacing
of resonances in different sequences as the dominant overlap
effect.  In addition, both codes stop the recursion of
Eq.~\ref{recursA} at $A_t=B_t$.  Thus,

\begin{equation}
  I_{0x} = \sum_s B_{xs} \Bigl(\,1-\sum_{s'\neq s} B_{ts'}\Bigr)\,\,,
\end{equation}

\noindent
and

\begin{equation}
  I_{1t} = \sum_s D_{ts}\Bigl(\,1-\sum_{s'\neq s}B_{ts'}\Bigr)^2\,\,.
\label{I1t}
\end{equation}

\noindent
The equations for the effective cross sections in the MC2/ETOX
approximation become

\begin{equation}
  \overline{\sigma}_{0x} = b_x + \frac{\displaystyle\overline{\sigma}
   \sum_s B_{xs} \Bigl(\,1-\sum_{s'\neq s}B_{ts'}\Bigr)}
    {\displaystyle 1-\sum_s B_{ts} \Bigl(\,1-\sum_{s'\neq s}B_{ts'}\Bigr)}\,\,,
\label{sb0x}
\end{equation}

\noindent
and

\begin{equation}
  \overline{\sigma}_{1t} = b_t + \frac{\displaystyle\overline{\sigma}
   \sum_s D_{ts} \Bigl(\,1-\sum_{s'\neq s}B_{ts'}\Bigr)^2}
    {\displaystyle 1-\sum_s B_{ts} \Bigl(\,1-\sum_{s'\neq s}B_{ts'}\Bigr)
     -\sum_s D_{ts}\bigl(\,1-\sum_{s'\neq s}B_{ts'}\Bigr)^2}\,\,,
\end{equation}

\noindent
or

\begin{equation}
  \overline{\sigma}_{1t} = \overline{\sigma}\left[
     \frac{\displaystyle 1-\sum_s B_{ts} \Bigl(\,1-\sum_{s'\neq s}B_{ts'}\Bigr)}
    {\displaystyle 1-\sum_s B_{ts} \Bigl(\,1-\sum_{s'\neq s}B_{ts'}\Bigr)
     -\sum_s D_{ts}\Bigl(\,1-\sum_{s'\neq s}B_{ts'}\Bigr)^2}
     \right]-\sigma_0 \,\,.
\label{sb1t}
\end{equation}

\noindent
These are the equations that are used in the UNRESR module of NJOY.
Note that the equation in the ETOX code and report corresponding
to Eq.~\ref{sb1t} is incorrect.  The following equation was
used in the ETOX code:

\begin{equation}
  \overline{\sigma}_{1t} = \overline{\sigma}\left[
     \frac{\displaystyle 1-\sum_s B_{ts} \Bigl(\,1-\sum_{s'\neq s}B_{ts'}\Bigr)}
    {\displaystyle 1-\sum_s C_{ts} \Bigl(\,1-\sum_{s'\neq s}C_{ts'}\Bigr)}
     \right]-\sigma_0 \,\,,
\end{equation}

\noindent
with $C_{ts}=B_{ts}+D_{ts}$.


\paragraph{The MC2-2 Approximation}
The MC2-2 code\index{MC2-2} includes the in-sequence overlap
corrections, which the authors found to be more important than previously
thought.  It uses additional approximations to obtain the equivalent of

\begin{equation}
  \overline{\sigma}_{0x} = b_x + \overline{\sigma}
    \sum_s \frac{B_{xs}-V_{0xs}}{1-B_{ts}+V_{0ts}}\,\,.
\end{equation}

\noindent
The additional approximations used are

\begin{enumerate}
\item Set $A_{ts}=B_{ts}-V_{0ts}$ (first-order sequence-sequence
overlap),
\item Neglect the factor $(1-\sum_{s'\neq s}A_{ts'})$ in the denominator, and
\item Use the approximation $1-\sum_if_i\approx\prod_i(1-f_i)$
on the numerator and denominator.
\end{enumerate}

These simplifications result in a loss of accuracy for the
sequence-sequence overlap correction at relatively low values of
$\sigma_0$.  The $\overline{\sigma}_{1t}$ term is not calculated.


\paragraph{The UXSR Approximation}
The experimental UXSR\index{UXSR} module was developed at Oak Ridge
\index{Oak Ridge National Laboratory!ORNL} (with some contributions from
LANL\index{Los Alamos National Laboratory!LANL})
based on coding from the Argonne National Laboratory (ANL)\index{Argonne
National Laboratory!ANL}
in an attempt to include the sophisticated in-sequence overlap corrections
from MC2-2 without approximating the sequence-sequence corrections
so badly.  It also implemented a calculation of the current-weighted
total cross section, which was omitted in MC2-2.  The additional
cost of using the full expressions for Eqs.~\ref{Izero} and \ref{Ione}
is minimal, and effective cross sections can be computed for
lower values of $\sigma_0$ when in-sequence overlap is
small ({\it e.g.,} $^{238}$U).

Now that expressions have been chosen for computing the cross
sections in terms of the isolated-resonance integrals, it is
necessary to select an efficient numerical method for computing
them.  The resonant parts of the cross sections are given by

\begin{equation}
  \sigma_{xsr}(E{-}E_{sr}) = \left[ \sigma_m\frac{\Gamma_x}{\Gamma}
    \psi(\theta,X) \right]_{sr}\,\,,
\end{equation}

\noindent
and

\begin{equation}
   \sigma_{tsr}(E{-}E_{sr})=\left[\, \sigma_m\{\cos 2\phi_\ell
     \,\psi(\theta,X) + \sin 2\phi_\ell\,\chi(\theta,X)\}\,\right]_{sr}\,\,,
\end{equation}

\noindent
where $x$ takes on the values $\gamma$, $f$, or $c$ for capture,
fission, or competition, and

\begin{equation}
   \sigma_m=\frac{4\pi g_J}{k^2}\frac{\Gamma_n}{\Gamma}\,\,,
\end{equation}

\begin{equation}
   \theta=\sqrt{\frac{A}{4kTE_0}}\,\Gamma\,\,,
\end{equation}

\begin{equation}
   X=\frac{2(E-E_0)}{\Gamma}\,\,,
\end{equation}

\begin{equation}
   g_J=\frac{2J+1}{2(2I+1)}\,\,,\;\hbox{and}
\end{equation}

\begin{equation}
   k = 2.196771\times 10^{-3}\frac{A}{1+A}\sqrt{E}\,\,.
\end{equation}

\noindent
The functions $\psi$ and $\chi$ are the symmetric and antisymmetric
components of the broadened resonance line shape:
\index{$\psi\chi$ broadening}

\begin{equation}
   \psi(\theta,X) = \frac{\theta\sqrt{\pi}}{2}
    {\rm Re}W\left(\frac{\theta X}{2},\frac{\theta}{2}\right)\,\,,
\end{equation}

\noindent
and

\begin{equation}
   \chi(\theta,X) = \theta\sqrt{\pi} {\rm Im}W\left(
      \frac{\theta X}{2},\frac{\theta}{2}\right)\,\,,
\end{equation}

\noindent
where

\begin{equation}
   W(x,y) = {\rm exp}[-(x+iy)^2]\,{\rm erfc}[-i(x+iy)]
\end{equation}

\noindent
is the complex probability integral.  The methods for computing
$\psi$ and $\chi$ are well known (see \cword{quikw}).

The first integral needed is

\begin{eqnarray}
   B_{xs} &=& \frac{1}{D_s}\int P(\alpha)\int
   \frac{\sigma_{xs\alpha}(\xi)}
        {\overline{\sigma}+\sigma_{ts\alpha}(\xi)}
           \,d\xi\,d\alpha \nonumber\\
   &=& \frac{1}{D_s}\int P(\alpha)\int
      \frac{\sigma_m (\Gamma_x/\Gamma)\psi(\theta,X)}
       {\overline{\sigma}+\sigma_m\{\cos 2\phi_\ell\, \psi(\theta,X)
           +\sin 2\phi_\ell\,\chi(\theta,X)\}}
            \,d\xi\,d\alpha \nonumber\\
   &=& \frac{1}{D_s}\int P(\alpha) \frac{\Gamma_x}{2\cos 2\phi_\ell}
       \int \frac{\psi(\theta,X)}
       {\beta+\psi(\theta,X)+\tan 2\phi_\ell\,\chi(\theta,X)}
        \,dX\,d\alpha\,\,,
\end{eqnarray}

\noindent
where

\begin{equation}
   \beta=\frac{\overline{\sigma}}{\sigma_m\cos 2\phi_\ell}\,\,.
\end{equation}

The second integral needed is

\begin{eqnarray}
   B_{ts} &=& \frac{1}{D_s}\int P(\alpha)
       \int\frac{\sigma_{ts\alpha}(\xi)}
        {\overline{\sigma}+\sigma_{ts\alpha}\xi)}
          \,d\xi\,d\alpha \nonumber\\
  &=& \frac{1}{D_s}\int P(\alpha)\,\frac{\Gamma}{2}\int
      \frac{\psi(\theta,X)+\tan 2\phi_\ell\,\chi(\theta,X)}
      {\beta+\psi(\theta,X)+\tan 2\phi_\ell\,\chi(\theta,X)}
      \,dX\,d\alpha\,\,.
\end{eqnarray}

Both of these integrals can be expressed in terms of the basic
$J$ integral:

\begin{eqnarray}
   B_{xs} &=& \frac{1}{D_s}\int P(\alpha)\frac{\Gamma}
     {\cos 2\phi_\ell}\,J(\beta,\theta,\tan2\phi_\ell,0)\,d\alpha
        \,\,,\;\hbox{and} \nonumber\\
   B_{ts} &=& \frac{1}{D_s}\int P(\alpha)\,\Gamma\,
       J(\beta,\theta,\tan 2\phi_\ell,\tan 2\phi_\ell)\,d\alpha\,\,,
\end{eqnarray}

\noindent
where

\begin{equation}
  J(\beta,\theta,a,b)=\frac{1}{2}\int_{-\infty}^\infty
   \frac{\psi(\theta,X)+b\,\chi(\theta,X)}
    {\beta+\psi(\theta,X)+a\,\chi(\theta,X)}\,dX\,\,.
\end{equation}

\noindent
The $D$ integral can be handled in the same way, but only total
reaction is required.

\begin{eqnarray}
   D_{ts} &=& \frac{1}{D_s}\int P(\alpha)\int
    \frac{\overline{\sigma}\sigma_{ts\alpha}(\xi)}
     {[\,\overline{\sigma}+\sigma_{ts\alpha}(\xi)\,]^2}
     \,d\xi\,d\alpha \nonumber\\
  &=& \frac{1}{D_s}\int P(\alpha)\,\frac{\Gamma}{2}\int
     \frac{\beta\psi(\theta,X)+\tan 2\phi_\ell\,\chi(\theta,X)}
     {[\,\beta+\psi(\theta,X)+\tan 2\phi_\ell\,\chi(\theta,X)\,]^2}
      \,dX\,d\alpha \nonumber\\
  &=& \frac{1}{D_s}\int P(\alpha)\,\Gamma\, K(\beta,\theta,\tan 2\phi_\ell,
     \tan 2\phi_\ell)\,d\alpha\,\,,
\end{eqnarray}

\noindent
where

\begin{equation}
   K(\beta,\theta,a,b)=\frac{1}{2}\int_{-\infty}^\infty
   \frac{\beta\,[\,\psi(\theta,X)+b\,\chi(\theta,X)\,]}
     {[\,\beta+\psi(\theta,X)+a\,\chi(\theta,X)\,]^2}\,dX\,\,.
\end{equation}

A method for computing $J$, including the interference effects, has
been developed by Hwang for MC2-2\cite{MC22}.  However,
this method was not available in the days when MC2 and ETOX
were developed.  Therefore, UNRESR uses only
$J(\beta,\theta,0,0)$ and $K(\beta,\theta,0,0)$
in computing the isolated-resonance fluctuation integrals.
A direct integration is used over most of the $X$ range, but
the part of the integral arising from large $X$ is handled
using analytic integrations of the asymptotic forms of
the arguments (see \cword{ajku}\index{ajku@{\ty ajku}}).

The final step is to do the n-fold integration over the probability
distributions for the resonance widths.  This integration has
been abbreviated as a single integration over $\alpha$ in the
above equations.  The method used was originally developed for
MC2-2 and is based on Gauss-Jacobi quadratures.
\index{Gauss-Jacobi quadrature} A set of 10
quadrature points and weights is provided for each of the
$\chi^2$ probability distributions with 1 through 4 degrees of
freedom.  These quadratures convert the n-fold integral into an
n-fold summation.  The value of $n$ can be as large as 4 when
$\Gamma_n$, $\Gamma_f$, $\Gamma_\gamma$, and $\Gamma_c$
(competitive width) are all present.

Although UNRESR neglects the effects of overlap between resonances in
the same spin sequence and the effects of interference in the elastic
and total cross sections, it still gives reasonable results for the
background cross section values needed for most practical problems.
Modern evaluations are steadily reducing the need for accurate
unresolved calculations by extending the resolved resonance range to
higher and higher energies.  Ultimately, UNRESR should be upgraded to
use the UXSR\index{UXSR} approach.  Another alternative is to generate
self-shielded effective cross sections from ladders of resonances
chosen statistically (see \hyperlink{sPURRhy}{PURR}\index{PURR}).  This
avoids many of the approximtions in the overlap corrections.

In NJOY2016, running the \hyperlink{sPURRhy}{PURR} module
after UNRESR overwrites the UNRESR output with the
\hyperlink{sPURRhy}{PURR} results.  In fact, UNRESR can be
omitted from the processing stream.  To use UNRESR results,
either omit \hyperlink{sPURRhy}{PURR} from the processing
stream or run it before running UNRESR.

\subsection{Implementation}
\label{ssUNRESR_implementation}

In implementing this theory in UNRESR, there are special considerations
involving the choice of an incident energy grid, what to do if the
unresolved range overlaps the resolved range or the range of smooth
cross sections, the choice of the $\sigma_0$ grid, how to interpolate
on $\sigma_0$, and how to communicate the results to other modules.

\paragraph{Choice of Energy Grid.}\index{unresolved energy grid}
The same logic is used to choose the incident energy grid in UNRESR
and \hyperlink{sRECONRhy}{RECONR}.  It is complicated, because
of the several different
representations available for unresolved data, and because of the
existence of evaluations that have been carried over from previous
versions of ENDF/B or ENDF/B-VII evaluations with inadequate
energy grids.  Even many modern evaluations have inadequate energy
grids.

For evaluations that give energy-independent unresolved-resonance
parameters, there is still an energy dependence to the cross sections.
Because this dependence is normally somewhere between constant and
a $1/v$ law, a fairly coarse grid with about 13 points per decade
should be sufficient to allow the cross sections to be computed
reliably using linear-linear interpolation.

If the evaluation uses energy-dependent parameters, the normal rule
would be to use the energies that were provided by the evaluator
and to obtain intermediate cross sections by interpolation.  Unfortunately,
some of the evaluations carried over from earlier days contain some
energy intervals that are quite large (for example, steps by factors
of 10).  The evaluators for these isotopes assumed that the user
would use parameter interpolation and compute the cross sections
at a number of intermediate energies in these long steps.  Even
some newer evaluations contain large jumps in the energy grid.  UNRESR
will detect such evaluations and add additional energy points in the
large energy steps using an algorithm similar to the one used for
the cases with energy-independent parameters.  For NJOY2016,
large jumps in the energy grid are any with step ratios greater
than \cword{wide}\index{wide@{\ty wide}}, which is currently set
to 1.26.

The final energy grid can be observed by scanning the printed
output from UNRESR.


\paragraph{Resolved-Unresolved Overlap.}\index{resolved-unresolved overlap}
Elemental evaluations include separate energy ranges in MF2/MT151
for each of the isotopes of the element, and these energy ranges
do not have to be the same for each isotope.  This means that the
lower end of an unresolved range may overlap the resolved
range from another isotope, and the upper end of the unresolved
range for an isotope can overlap the smooth range of another isotope.
These overlap regions are detected by UNRESR as the resonance data
are read in, and they are marked by making the sign of the incident
energy value negative.


\paragraph{Choosing a $\sigma_0$ Grid.}\index{$\sigma_0$!$\sigma_0$
interpolation}
There are two factors to consider, namely, choosing values that
will represent the shape adequately, and limiting the range of
$\sigma_0$ to the region where the theory is valid.  The
$\sigma_x(\sigma_0)$ curves start out decreasing from the
infinite dilution value as $1/\sigma_0$ as $\sigma_0$ decreases
from infinity ($1\times 10^{10}$ in the code).  The curve eventually
goes through an inflection point at some characteristic value
of $\sigma_0$, becomes concave upward, and approaches a limiting
value at small $\sigma_0$ that is smaller than the
infinite-dilution value.  Decade steps are often used, but the
user should try to select values that include the inflection point
and not waste values on the $1/\sigma_0$ region.  Half-decade
values are useful near the inflection point ({\it e.g.,} 100,
300, 1000 for $^{235}$U).  The grid interval chosen should be
consistent with the interpolation method used (see below).

Choosing the lower limit for $\sigma_0$\index{unresolved $\sigma_0$ range}
is a more difficult problem.  As shown in the theory section
(\ref{ssUNRESR_theory}),
resonance overlap effects are developed as a series in $1/\sigma_0$,
and the series is truncated after only one step of recursion
in Eq.~\ref{recursA}.  This means that the overlap correction
should be most accurate for large $\sigma_0$ and gradually get worse
as $\sigma_0$ decreases.  Experience shows that the correction can
actually get large enough to produce negative cross sections for
small $\sigma_0$.  (This problem can also show up as a failure in
interpolation when a log scheme has been selected.)  For isotopes that
have relatively narrow resonances spaced relatively widely, such as
$^{238}$U, UNRESR gives reasonable results to $\sigma_0$ values
as low as 0.1.  For materials with stronger overlap, such as
$^{235}$U, a lower limit around 100 is more reasonable.  A few
of the heavy actinide evaluations have been seen to break
down for $\sigma_0$ values lower than 1000.  This problem is not
too serious in practice.  The fertile materials, which appear
in large concentrations in reactors, allow the necessary small
values of $\sigma_0$.  The fissile materials have to be more
dilute, and the larger $\sigma_0$ limit needed for them is not
usually a problem.

The UXSR\index{UXSR} approximation discussed above allows one to reach
somewhat smaller $\sigma_0$ values.

\paragraph{Interpolating on $\sigma_0$.}\index{$\sigma_0$!$\sigma_0$
interpolation}
It turns out that these functions are difficult to interpolate because
they have a limited radius of convergence.   Although approximate
schemes have been developed based on using functions of similar
shape such as the tanh function\cite{Kidman}\index{Kidman}, better
results can be obtained by using different interpolation schemes for the
low- and high-$\sigma_0$ ranges.  The TRANSX-CTR
code\cite{TRANSX}\index{TRANSX-CTR} used interpolation in
$1/\sigma_0$ for high $\sigma_0$, Lagrangian interpolation of
ln$\sigma_x$ {\it vs} ln$\sigma_0$, for intermediate values, and
a $\sigma_0^2$ extrapolation for very low $\sigma_0$.  Unfortunately,
schemes like this sometimes give ridiculous interpolation excursions
when the polynomials are not suitable.  Therefore,
TRANSX-2\cite{TRANSX2}\index{TRANSX-2} has had to revert to using
simple linear interpolation, which is always bounded and
predictable, but which requires relatively fine $\sigma_0$ grids.

\paragraph{Communicating Results to Other Modules.}  NJOY tries
to use ENDF-like files for all communications between the different
calculational modules.  Because the unresolved effective cross
sections were originally derived from the resonance parameters in
File 2, it seemed reasonable to create a new section in File 2
to carry the unresolved cross sections onto other modules, and
a special MT number of 152\index{MT!MT152} was selected for this purpose.
\hyperlink{sRECONRhy}{RECONR}\index{RECONR} creates
an MT152 that contains only the
infinitely-dilute unresolved cross sections.  UNRESR overwrites it
with self-shielded unresolved cross
sections.  \hyperlink{sGROUPRhy}{GROUPR}\index{GROUPR}
can then use these cross sections in its calculation of the multigroup
constants.  The format used for MT152 is given below using the
conventional ENDF style.

\small
\begin{ccode}

[MAT,2,152/ ZA, AWR, LSSF, 0, 0, INTUNR ] HEAD
[MAT,2,152/ TEMZ, 0, NREAC, NSIGZ, NW, NUNR/
            SIGZ(1), SIGZ(2),...,SIGZ(NSIGZ),
            EUNR(1),
            SIGU(1,1,1), SIGU(1,2,1),...,SIGU(1,NSIGZ,1),
            SIGU(2,1,1),...
            ...
            SIGU(NREAC,1,1),...,SIGU(NREAC,NSIGZ,1),
            EUNR(2),...
              <continue for energies through EUNR(NUNR)>
            ...SIGU(NREAC,NSIGZ,NUNR) ] LIST

\end{ccode}
\normalsize

\noindent
where \cword{NREAC} is always 5 (for the total, elastic,
fission, capture, and current-weighted total reactions, in that order),
\cword{NSIGZ} is the number of $\sigma_0$ values, \cword{NUNR} is
the number of unresolved energy grid points, and \cword{NW} is
given by

\small
\begin{ccode}

   NW=NSIGZ+NUNR*(1+5*NSIGZ)

\end{ccode}
\normalsize

\subsection{User Input}
\label{ssUNRESR_inp}

The following summary of the user input instructions was copied
from the comment cards at the beginning of the UNRESR module
in the NJOY2016 source file.
\index{UNRESR!UNRESR input}
\index{input!UNRESR}

\small
\begin{ccode}

   !---input specifications (free format)---------------------------
   !
   ! card 1
   !   nendf   unit for endf tape
   !   nin     unit for input pendf tape
   !   nout    unit for output pendf tape
   ! card 2
   !   matd    material to be processed
   !   ntemp   no. of temperatures (default=1)
   !   nsigz   no. of sigma zeroes (default=1)
   !   iprint  print option (0=min, 1=max) (default=0)
   ! card 3
   !   temp    temperatures in Kelvin (including zero)
   ! card 4
   !   sigz    sigma zero values (including infinity)
   !       cards 2, 3, 4 must be input for each material desired
   !       matd=0/ terminates execution of unresr.
   !
   !--------------------------------------------------------------------

\end{ccode}
\normalsize

Card 1, as usual, specifies the input and output units for the
module.  The input PENDF file on \cword{nin} must have been
processed through \hyperlink{sRECONRhy}{RECONR}
and \hyperlink{sBROADRhy}{BROADR}.  It will contain default
versions of the special unresolved section with MF=2 and MT=152 that
gives the infinitely-dilute unresolved cross sections for each
temperature.  The output PENDF file \cword{nout} will contain
revised sections MF2, MT152 that give effective cross sections
{\it vs} $\sigma_0$ for all temperatures.

Card 2 is used to specify the material desired (\cword{matd}), the
number of temperatures and background cross sections desired
(\cword{ntemp} and \cword{nsigz}), and the print option
(\cword{iprint}).  The actual temperature and $\sigma_0$ values
are given on Cards 3 and 4.  Temperatures should be chosen to
be adequate for the planned applications.  The temperature
dependence of the effective cross sections is usually monotonic
and fairly smooth.  Polynomial interpolation schemes using
$T$ are often used, and roughly uniform spacing for
the temperature grid (or spacing that increases modestly as
$T$ increases) is usually suitable.

The choice of a grid for $\sigma_0$ is more difficult.  The
curves of cross section versus $\sigma_0$ have an inflection
point, and it is important to choose the grid to represent the
inflection point fairly well.  A log spacing for $\sigma_0$ is
recommended.  Very small values of $\sigma_0$ should not be used.
These considerations are discussed in more detail in the
``Implementation'' section (\ref{ssUNRESR_implementation}) above.
Unfortunately, the best choice for a grid can only be found by trial and error.

\subsection{Output Example}
\label{ssUNRESR_output}

The following portion of UNRESR output is for $^{238}$U from
ENDF/B-VII.0.
\index{UNRESR!UNRESR output}

\small
\begin{ccode}

 unresr...calculation of unresolved resonance cross sections          494.4s

                                                        storage   8/   20000

 unit for input endf/b tape ...........        -21
 unit for input pendf tape ............        -23
 unit for output pendf tape ...........        -24

 temperatures .........................  2.936E+02
                                         4.000E+02
                                          ...
 sigma zero values ....................  1.000E+10
                                         1.000E+03
                                         1.000E+02
                                         5.000E+01
                                         2.000E+01
                                         1.000E+01
                                         5.000E+00
                                         2.000E+00
                                         1.000E+00
                                         5.000E-01
 print option (0 min., 1 max.) ........          1

 mat = 9237    temp =  2.936E+02                                      494.4s
 energy =  2.0000E+04
   1.433E+01  1.428E+01  ... 1.311E+01  1.297E+01 1.291E+01  1.288E+01
   1.380E+01  1.375E+01  ... 1.264E+01  1.252E+01 1.247E+01  1.244E+01
   0.000E+00  0.000E+00  ... 0.000E+00  0.000E+00 0.000E+00  0.000E+00
   5.294E-01  5.280E-01  ... 4.654E-01  4.540E-01 4.491E-01  4.464E-01
   1.433E+01  1.424E+01  ... 1.246E+01  1.230E+01 1.223E+01  1.220E+01
 energy =  2.3000E+04
   1.414E+01  1.411E+01  ... 1.307E+01  1.293E+01 1.288E+01  1.285E+01
   1.364E+01  1.361E+01  ... 1.262E+01  1.250E+01 1.245E+01  1.242E+01
   0.000E+00  0.000E+00  ... 0.000E+00  0.000E+00 0.000E+00  0.000E+00
   4.962E-01  4.951E-01  ... 4.426E-01  4.325E-01 4.281E-01  4.257E-01
   1.414E+01  1.407E+01  ... 1.246E+01  1.229E+01 1.223E+01  1.219E+01
   ...
 energy =  1.4903E+05
   1.140E+01  1.140E+01  ... 1.124E+01  1.118E+01 1.115E+01  1.113E+01
   1.126E+01  1.126E+01  ... 1.110E+01  1.104E+01 1.101E+01  1.099E+01
   0.000E+00  0.000E+00  ... 0.000E+00  0.000E+00 0.000E+00  0.000E+00
   1.427E-01  1.427E-01  ... 1.406E-01  1.386E-01 1.374E-01  1.367E-01
   1.213E+01  1.212E+01  ... 1.174E+01  1.156E+01 1.147E+01  1.141E+01
 generated cross sections at  18 points                               494.9s

\end{ccode}
\normalsize

\noindent
The display of the cross section table for each energy has $\sigma_0$
reading across (in decreasing order) and reaction type reading down
(in the order of total, elastic, fission, capture, and current-weighted
total).  Four $\sigma_0$ values were removed from the table to make
it narrower for this report.

\subsection{Coding Details}
\label{ssUNRESR_details}

The main entry point for UNRESR is subroutine \cword{unresr}\index{unresr},
which is exported by module \cword{unresm}.\index{modules!unresm@{\ty unresm}}
The subroutine starts by reading in the user's input and output unit
numbers and opening the files that will be needed during the
UNRESR run.  After printing the introductory timer line, storage is
allocated for an array \cword{scr}, which will be used for reading
in ENDF records.  The default size of this array is \cword{maxscr}=1000,
which has proved sufficient for all evaluations tried so far.
UNRESR now prints out the user's unit numbers on the output
listing and calls \cword{uwtab}\index{uwtab@{\ty uwtab}} to prepare
internal tables used by \cword{uw}\index{uw@{\ty uw}} to compute the
real and imaginary parts of the complex probability
integral\index{complex probability integral}.

The next step is to read in the \cword{tapeid} records of the
ENDF and PENDF tapes.  The loop over materials starts at
statement number 110 by reading in the user's values for the
ENDF MAT number, number of temperatures \cword{ntemp}, number
of $\sigma_0$ values \cword{nsigz}, and print flag \cword{iprint}.
If this is not the end of the material loop (\cword{mat}=0),
the actual values of the temperatures and background cross
sections are read into the arrays \cword{temp} and \cword{sigz}.
The input quantities are echoed to the output listing to help
detect input errors.

The code then begins a loop over the requested temperatures.  It
writes the current values of material ID, temperature, and time
on the output listing. It then reads the resonance parameters
from the section with MT=151 in File 2 of the ENDF tape
using \cword{rdunf2}. The arrays \cword{eunr} with length
\cword{maxeunr}=150 and \cword{arry} with length
with length \cword{maxarry}=10000 are used to
store these data.  Next, it reads the background cross sections
from File 3 for each of the resonance reactions using
\cword{rdunf3}\index{rdunf3@{\ty rdunf3}}.  Here, \cword{sb}
is used to store the data.  The array \cword{b} is allocated
with sufficient length to build the output record to be
written in MT=152\index{MT!MT152} on the new PENDF file.

The next loop is over all the energy grid points at this
temperature.  The grid points were determined in
\cword{rdunf2}\index{rdunf2@{\ty rdunf2}}, and the \cword{nunr}
points are stored in the array \cword{eunr}.
The background cross sections are stored in an array
\cword{sb(ie,ix)}.  The energy index takes on \cword{nunr} values,
and the reaction index \cword{ix} takes on four values.  The
calculation of the actual effective cross sections takes place
in \cword{unresl}.  The results for each energy appear in the
array \cword{sigu(ix,is)}, where \cword{is} takes on
\cword{nsigz} values.  Each \cword{unresl} array is stored
into the accumulating output array \cword{b} and printed on
the output listing.

At this point, UNRESR checks to see if there is a previous
version of MT152\index{MT!MT152} on the PENDF tape.  If so, these new
data will replace it.  If not, a new section with MT=152 will be
created.  In either case, a new section MT451 in File 1 is generated
containing the current temperature and the correct entry in
the directory for the PENDF tape.  Finally, the new MT152 for
this temperature is copied onto the output file from the
\cword{b} array, and the rest of the contents of this temperature
on the old PENDF file are copied to the new PENDF file.
After writing a report on the number of resonance points generated
and the amount of computer time used, UNRESR branches back to
continue the temperature loop.  When the last temperature has
been processed, the code closes the files, writes a final report
on the output listing, and terminates.

Note that UNRESR takes special branches for materials with no
resonance parameters or materials with no unresolved parameters.
Therefore, the user can freely request an UNRESR run even when there
are no unresolved resonance data present on the ENDF tape.  UNRESR
simply copies \cword{nin} to \cword{nout} in this case.

The subroutine \cword{rdunf2}\index{rdunf2@{\ty rdunf2}} is used
to read in the unresolved resonance parameters from File 2 of
the input ENDF tape.  It is very similar to the corresponding
coding in \hyperlink{sRECONRhy}{RECONR}.  The array
\cword{scr} is used to read in the
ENDF record, the resonance parameter data are stored in the
array \cword{arry}, and the final grid of energy values is
stored in \cword{eunr}.  Note that \cword{rdunf2} will add
extra energy nodes for evaluations with energy-independent
parameters or for energy-dependent evaluations that have energy
steps larger than the factor \cword{wide}\index{wide@{\ty wide}},
which is currently set to 1.26.  The subroutine
also discovers resolved-unresolved or unresolved-smooth overlap,
flags those energy values, and prints messages to the user on
the output listing.

Subroutine \cword{ilist}\index{ilist@{\ty ilist}} is used to insert
a new energy into a list of energies in ascending order.  It is used
to manage the accumulation of the grid of energy nodes.

Subroutine \cword{rdunf3}\index{rdunf3@{\ty rdunf3}} is used to read
in the background cross sections in the unresolved range from File 3.
The unresolved energy grid determined by
\cword{rdunf2}\index{rdunf2@{\ty rdunf2}} is used
for the background cross sections.

The main calculation of the effective cross sections for the
unresolved range is performed in subroutine
\cword{unresl}\index{unresl@{\ty unresl}}.
The calculation is done in two passes: first, the potential
scattering cross section is computed; second, the unresolved
cross sections are computed.  The passes are controlled by
the parameter \cword{ispot}.  In both cases, resonance parameter
data stored in \cword{arry} by \cword{rdunf2} are used.  The
coding is similar to that used in \hyperlink{sRECONRhy}{RECONR} down
to the point
where the ETOX statistical averages are computed.  The three
loops over \cword{kf}, \cword{kn}, and \cword{kl} carry out
the n-fold quadrature represented as integrals over $\alpha$
in the text.  They account for variations in the fission width,
neutron width, and competitive width.  The capture width is
assumed to be constant.  Note that
\cword{ajku}\index{ajku@{\ty ajku}} is called in
the innermost loop to compute the $J$ and $K$ integrals in
\cword{xj} and \cword{xk}, respectively.  The $K$ integral
returned by this routine is actually $J+K$ in our notation.
The final quantities are computed in the loops over
\cword{itp} and \cword{is0}.  Note that \cword{tk} is
equivalent to $B_{ts}+D_{ts}$ in our notation.  Similarly,
\cword{tl} is equivalent to $B_{xs}$, and \cword{tj} is
equivalent to $B_{ts}$.  The last step in this subroutine is
to compute the average cross sections by summing over spin
sequences.  The loop over \cword{ks} computes \cword{xj} as
$\sum_s B_{ts}$ and \cword{xk} as $\sum_s C_{ts}$.  With these
quantities available, it is easy to finish the calculation of
the effective cross sections as given by Eq.~\ref{sb0x}
and Eq.~\ref{sb1t}.

The subroutines \cword{uunfac}\index{uunfac@{\ty uunfac}},
\cword{intrf}\index{intrf@{\ty intrf}}, and
\cword{intr}\index{intr@{\ty intr}}
are similar to corresponding routines
from \hyperlink{sRECONRhy}{RECONR}\index{RECONR}
and don't require further discussion here.  Subroutine
\cword{ajku}\index{ajku@{\ty ajku}} is used to compute
the $J$ and $K$ integrals without interference corrections.
The subroutines \cword{quikw}\index{quikw@{\ty quikw}},
\cword{uwtab}\index{uwtab@{\ty uwtab}}, and
\cword{uw}\index{uw@{\ty uw}} implement a package for computing
the complex probability integral efficiently. It was originally
developed at ANL\index{Argonne National Laboratory!ANL} for the
MC2\index{MC2} code.
When it is used, a pair of 62 $\times$ 62 tables for the real and
imaginary parts of the complex probability integral
\index{complex probability integral} are precomputed using
\cword{uwtab} and \cword{uw}.  Values of W for small arguments
are obtained by interpolating in these precomputed tables.
Values of W for larger arguments are obtained using various
asymptotic formulas.

\subsection{Error Messages}
\label{ssUNRESR_msg}

\begin{description}
\begin{singlespace}

\item[\cword{error in unresr***mode conversion between nin and nout}] ~\par
   Input and output files must both be in ASCII mode (positive
   unit numbers), or they must both be in binary mode (negative
   unit numbers).

\item[\cword{error in unresr***storage exceeded}] ~\par
    There is not enough room in the \cword{b} array.  Increase
    \cword{nb}, which currently is 5000.

\item[\cword{error in rdunf2***storage exceeded}] ~\par
    There is not enough room in \cword{arry}.  Increase the
    value of \cword{maxarry}, which currently is 10000.


\item[\cword{error in unresl***storage exceeded}] ~\par
    The code is currently limited to 50 spin sequences.  For
    more spin sequences, it will be necessary to increase the
    dimensions of several arrays in this subroutine.

\end{singlespace}
\end{description}

\cleardoublepage

