\section{NJOY Maintenance and Testing}
\label{sMandT}

There are several components of the NJOY Quality Assurance
\index{Quality Assurance!QA}system:  a detailed code manual
(this report), version control software (\texttt{git})
\index{git@{\ty git}} that
keeps track of all changes made to the code, a suite of standard
test problems to verify installation, and extensive application
of the code to find special cases that lead to failures.

\subsection{Code Maintenance with GIT}
\label{ssMandT_GIT}

Begininng with NJOY2016, \texttt{git} is used for software version
control. This is a significant departure from UPD\index{UPD}\cite{UPD},
the local version control program used for past NJOY
versions.  \texttt{git} version control software is used virtually
everywhere modern software development is utilized and is available
on platforms using Windows, Linux, and Mac operating
systems. \texttt{git} will keep track of the changes made to
all files it is tracking.

This manual is not a proper forum to provide a tutorial on using
\texttt{git}. A brief tutorial is available at
\url{https://try.github.io}.  This introduction provides the reader
with enough understanding to be able to perform basic \texttt{git}
operations.  For additional information, we recommend users check the
official \texttt{git} website, \url{https://git-scm.com}, or other
online resources (e.g., \emph{Pro Git}\cite{GIT}).

\subsection{Standard Test Problems}
\label{ssMandT_testprob}

Another important part of the NJOY revision control procedure is
the set of standard test problems used to validate each new
\index{testing!test problems}
version.  This kind of systematic testing is also a key part of
any QA program.  \index{Quality Assurance!QA} The
NJOY test problems also act as examples in helping new users
to operate the code system.  Brief descriptions of the current sec
of test problems follow.  See  the following sections for details.

\begin{description}
\begin{singlespace}
\item[Problem 1:]  Process one ENDF/B-V isotope through pointwise
   and multigroup modules.  It tests heating and damage
   calculations, thermal calculations for free-gas carbon and
   carbon bound in graphite, and multigroup averaging.  The
   full PENDF tape is included in the test comparisons.  ENDF/B-V
   Tape 511 and ENDF/B-III thermal tape T322 are provided in the
   NJOY2016 package.
\item[Problem 2:]  Process one ENDF/B-IV isotope for a practical
   CCCC library.  It tests resonance reconstruction,
   Doppler-broadening to several temperatures, self-shielded unresolved
   cross sections, self-shielded multigroup cross sections, and the
   CCCC files ISOTXS, BRKOXS, and DLAYXS.  Tape 404 is provided in the
   NJOY2016 package.
\item[Problem 3:]  Process photon interaction cross sections
   into DTF and MATXS formats.  The problem tests photoatomic cross
   section linearization in \hyperlink{sRECONRhy}{RECONR},
   multigroup averaging in \hyperlink{sGAMINRhy}{GAMINR}, and
   output formatting in \hyperlink{sDTFRhy}{DTFR} and
   \hyperlink{sMATXSRhy}{MATXSR}.  The
   DLC7E library is provided in the NJOY2016 package.
\item[Problem 4:]  \hyperlink{sERRORRhy}{ERRORR} is tested,
   including the calculation of
   covariances for fission $\bar{\nu}$.  Tape 511 is used.
\item[Problem 5:]  This run tests \hyperlink{sCOVRhy}{COVR},
   including the plotting
   capability.  Tape 511 is used. This calculation produces a
   large number of covariance graphs.
\item[Problem 6:]  Includes a number of 2-D sample problems for
   \hyperlink{sPLOTRhy}{PLOTR}, and one 3-D case.  Plots with
   special characters,
   error bars, curve tags, and legend blocks are demonstrated.
\item[Problem 7:]  Prepares an ACE-format library for a fissionable
   material.
\item[Problem 8:] Checks the processing of a typical ENDF/B-VI
   material using Reich-Moore resonances and File 6 for energy-angle
   distributions through PENDF production and
   \hyperlink{sACERhy}{ACER} formatting.
\item[Problem 9:] Demonstrates the use of
   \hyperlink{sLEAPRhy}{LEAPR} to generate
   a scattering kernel for water.  The $\alpha$ and $\beta$ ranges
   have been reduced to make the case run faster with less output.
\item[Problem 10:] The production of unresolved resonance probability
   tables for MCNP is demonstrated.  \hyperlink{sUNRESRhy}{UNRESR}
   and \hyperlink{sPURRhy}{PURR} are both run to
   allow comparisons of the Bondarenko results, and then
   \hyperlink{sACERhy}{ACER} is
   run to format the results for MCNP.
\item[Problem 11:] Demonstrates the production of a library for the
   WIMS reactor lattice code using $^{238}$Pu from ENDF/B-V.  PENDF
   processing, \hyperlink{sGROUPRhy}{GROUPR}, and
   \hyperlink{sWIMSRhy}{WIMSR} are all included.
\item[Problem 12:]  Shows how to use \hyperlink{sGASPRhy}{GASPR}
   to generate gas-production
   data on the PENDF file, including color Postscript plots of the
   resulting cross sections.
\item[Problem 13:] Demonstrates the ``new'' MCNP formats and ACE
   plotting.
\item[Problem 14:] Shows how to prepare ACE incident proton
   data and demonstrates the charged-particle format.  The necessary
   evaluation is provided.
\item[Problem 15:]  Executes \hyperlink{sMODERhy}{MODER}/
   \hyperlink{sRECONRhy}{RECONR}/\hyperlink{sBROADRhy}{BROADR}/
   \hyperlink{sGROUPRhy}{GROUPR}/\hyperlink{sERRORRhy}{ERRORR}
   (once each for MF31, MF33 and MF34).  This job illustrates
   NJOY's ability
   to process uncertainty data for $\bar{\nu}$ (MF31), pointwise cross
   sections (MF33) and angular distributions (MF4/MT2 P$_1$ moment).
   The ``ENDF' input tape is the JENDL-3.3 $^{238}$U evaluation,
   demonstrating that non-ENDF libraries that conform to the ENDF-6
   format can be successfully processed by NJOY.  This input tape is
   provided in the NJOY2016 package.
\item[Problem 16:]  This test job is similar to Problem 15, but it
   omits the \hyperlink{sGROUPRhy}{GROUPR} module,
   demonstrating that uncertainty data
   (MF33 \& MF34) processing can proceed directly from PENDF input.
   We also append multiple \hyperlink{sCOVRhy}{COVR} and
   \hyperlink{sVIEWRhy}{VIEWR} inputs to this job to
   illustrate postscript plot generation for a user-specified set
   of cross sections (MF33) and automatic plot generation for MF34.
\item[Problem 17:]  This is the longest running job in the NJOY test
   suite, involving processing of $^{235,238}$U and $^{239}$Pu.  The
   job suite includes \hyperlink{sRECONRhy}{RECONR},
   \hyperlink{sBROADRhy}{BROADR}, and \hyperlink{sGROUPRhy}{GROUPR}
   for each nuclide, a \hyperlink{sMODERhy}{MODER} job to combine
   the GENDF files and ERRORR processing that includes correlations
   among the isotopes.  The necessary JENDL-3.3
   input files are provided in the NJOY2016 package.
\item[Problem 18:]  Execute \hyperlink{sMODERhy}{MODER}/
   \hyperlink{sRECONRhy}{RECONR}/\hyperlink{sBROADRhy}{BROADR}/
   \hyperlink{sGROUPRhy}{GROUPR}/\hyperlink{sERRORRhy}{ERRORR}/
   \hyperlink{sCOVRhy}{COVR} and \hyperlink{sVIEWRhy}{VIEWR}
   to process MF35 (spectrum) uncertainty data.  The multigroup
   energy boundaries used in \hyperlink{sGROUPRhy}{GROUPR}
   and \hyperlink{sERRORRhy}{ERRORR} match those
   used to define the uncertainty data on the input tape and allow
   for easy comparison of the processed output and the original data.
   The ``ENDF'' input file is a composite of ENDF/B-VII.0 $^{252}$Cf
   decay data (for MF5 \& MF35, MT18) and ENDF/B-VII.0 $^{252}$Cf
   neutron transport data (for all other MF/MT data) and given a
   dummy MATN of 9999.
\item[Problem 19:] Tests processing of a Reich-Moore evaluation
   ($^{241}$Pu) in an ACE file for MCNP.  An RM evaluation from ORNL
   is provided.
\item[Problem 20:] Tests processing of covariance data from
   Reich-Moore-Limited resonance parameters using an
   experimental $^{35}$Cl evaluation from ORNL that is included in
   the NJOY2016 package.
\end{singlespace}
\end{description}

\subsection{Test Problem 1}
\label{ssMandT_1}
\index{testing!Problem 01}

This problem demonstrates how to prepare data for natural carbon
as given on the ENDF/B-V ``Standards Tape'' and one of the ENDF/B-III
thermal tapes.  We've kept on using these old input libraries
for this test over all the versions of NJOY for consistency.
\index{testing!Problem 01}
\index{ENDF!ENDF/B-V}
\index{ENDF!ENDF standards tape}

The input cards for NJOY are listed below in the form of
a UNIX shell script.  We normally run this script in a subdirectory
called \cword{test}, and the  first few cards copy the
ENDF general-purpose and thermal data from their normal locations
in the next higher directory into the test subdirectory.  Note
that the data files are assigned the local names \cword{tape20}
and \cword{tape26}.  The \cword{cat} line starts a ``here'' file,
which continues down to the \cword{eof} line near the end of
the input.  The NJOY code is then run using this new input file,
and the output file and PENDF file are saved in the names
\index{PENDF} \cword{out01}, and \cword{pend01} for later comparisons
with previous runs.

\small
\begin{ccode}

echo 'NJOY Test Problem 1'
echo 'getting endf tape 511'
cp ../t511 tape20
echo 'getting thermal tape 322'
cp ../t322 tape26
echo 'running njoy'
ulimit -s 32768
cat>input <<EOF
 moder
 20 -21
 reconr
 -21 -22
 'pendf tape for c-nat from endf/b tape 511'/
 1306 3/
 .005/
 '6-c-nat from tape 511'/
 'processed by the njoy nuclear data processing system'/
 'see original endf/b-v tape for details of evaluation'/
 0/
 broadr
 -21 -22 -23
 1306 1/
 .005/
 300.
 0/
 heatr
 -21 -23 -22/
 1306 1/
 444
 thermr
 0 -22 -24
 0 1306 8 1 1 0 0 1 221 2
 300.
 .05 1.2/
 thermr
 26 -24 -23
 1065 1306 8 1 2 1 0 1 229 2
 300.
 .05 1.2/
 groupr
 -21 -23 0 -24
 1306 3 3 3 3 1 1 1
 'carbon in graphite'/
 300
 1.e10
 3 1 'total'/
 3 2 'elastic'/
 3 4 'inelastic'/
 3 51 'discrete inelastic'/
 3 -68 'continued'/
 3 91 'continuum inelastic'/
 3 102 'n,g'/
 3 103 '(n,p)'/
 3 104 '(n,d)'/
 3 107 '(n,a)'/
 3 221 'free thermal scattering'/
 3 229 'graphite inelastic thermal scattering'/
 3 230 'graphite elastic thermal scattering'/
 3 251 'mubar'/
 3 252 'xi'/
 3 253 'gamma'/
 3 301 'total heat production'/
 3 444 'total damage energy production'/
 6 2 'elastic'/
 6 51 'discrete inelastic'/
 6 -68 'continued'/
 6 91 'continuum inelastic'/
 6 221 'free thermal scattering'/
 6 229 'graphite inelastic thermal scattering'/
 6 230 'graphite elastic thermal scattering'/
 17 51 'inelastic gamma production'/
 16 102 'capture gamma production'/
 0/
 0/
 moder
 -23 25
 stop
EOF
../xnjoy<input
echo 'saving output and pendf files'
cp output out01
cp tape25 pend01

\end{ccode}
\normalsize

The first step is to run the \hyperlink{sMODERhy}{MODER}
module to convert the ASCII ENDF file to binary mode.  Using
binary mode will often cut the cost of running NJOY
jobs.  \hyperlink{sRECONRhy}{RECONR} is then used to linearize and
unionize the cross sections (no resonance reconstruction is
needed for carbon).  A tolerance of 0.5\% was requested for this
linearization.  The \hyperlink{sBROADRhy}{BROADR} module
is used to Doppler-broaden the carbon cross sections to
300K.  It is recommended that the same thinning and
reconstruction tolerances be used in
\hyperlink{sBROADRhy}{BROADR} as are used in
\hyperlink{sRECONRhy}{RECONR}.
\index{MODER}
\index{RECONR}
\index{BROADR}

This NJOY run supplements the original ENDF data with computed
values for heating, radiation damage, and thermal scattering.
The call to the \hyperlink{sHEATRhy}{HEATR} module
requests MT444 to get the damage cross section; heating
(MT301) is automatically provided.  The first of the two
\hyperlink{sTHERMRhy}{THERMR} runs generates thermal scattering data
for a carbon free gas at 300K (MT221).  The second
\hyperlink{sTHERMRhy}{THERMR} run generates
data for graphite (MT229).  The ``1'' in the sixth field
on the input card directs the module to use the $S(\alpha,\beta)$
data from MT1065 on \cword{tape26} in order to compute the inelastic
scattering cross section and scattering matrix, the next field of
``0'' requests $E,E',\mu$ ordering, and the following ``1'' directs
it to compute the coherent elastic scattering cross section using
the built-in parameters for graphite.
\index{HEATR}
\index{THERMR}
\index{graphite}

When the second \hyperlink{sTHERMRhy}{THERMR} run has
finished, \cword{tape23} contains the complete PENDF tape needed by
\hyperlink{sGROUPRhy}{GROUPR}.\index{GROUPR}  Multigroup
neutron reaction and photon production cross sections are computed
using the Los Alamos 30-group structure for neutrons, the Los Alamos
12-group structure for photons, and the CLAW weight function.
\index{30-group structure}
\index{12-group structure}
\index{CLAW weight function}
The scattering order is P$_3$.  Note that the long-input form
is used to specify the list of reactions to be processed.  Most
users now prefer the automatic input option.  The user has to
carefully look at the reactions available on the ENDF tape and
to consider the additional reactions added by
\hyperlink{sHEATRhy}{HEATR} and
\hyperlink{sTHERMRhy}{THERMR}.  Note especially the
inclusion of several gas production reactions
from the ENDF tape, the thermal reactions MT221 and MT229 from
the PENDF tape, and MT301 and MT444 as generated by
\hyperlink{sHEATRhy}{HEATR} from the PENDF tape.  The
output is easy to read, but remember that
groups are given in the order of increasing energy.

The thermal scattering reactions must also be requested in the
scattering matrix section (MFD=6).  MT230 was automatically
generated by \hyperlink{sTHERMRhy}{THERMR} as MT229+1
when coherent elastic scattering
for graphite was requested.

Photon production matrices are requested in the lines with
\cword{mfd=17} and \cword{mfd=16}.  Actually, the use of ``17''
to denote data given in MF13 on the ENDF tape is no longer
required by \hyperlink{sGROUPRhy}{GROUPR}.

As a last step, this problem runs \hyperlink{sMODERhy}{MODER}
once more to convert the binary PENDF tape to ASCII mode.


\subsection{Test Problem 2}
\label{ssMandT_2}
\index{testing!Problem 02}

The second problem supplements the first by adding resonance
reconstruction and output formatting.  The ENDF/B-IV material
$^{238}$Pu (MAT1050) was originally chosen for this problem because
it was freely available and its execution time was fairly small.
\index{ENDF!ENDF/B-IV}

\small
\begin{ccode}

echo 'NJOY Test Problem 2'
echo 'getting endf tape 404'
cp ../t404 tape20
echo 'running njoy'
cat>input <<EOF
 moder
 20 -21
 reconr
 -21 -22
 'pendf tape for pu-238 from endf/b-iv tape 404'/
 1050 3/
 .005/
 '94-pu-238 from endf/b tape t404'/
 'processed by the njoy nuclear data processing system'/
 'see original endf/b-iv tape for details of evaluation'/
 0/
 broadr
 -21 -22 -23
 1050 3 0 1 0/
 .005/
 300. 900. 2100.
 0/
 moder
 -23 33
 unresr
 -21 -23 -24
 1050 3 7 1
 300 900 2100
 1.e10 1.e5 1.e4 1000. 100. 10. 1
 0/
 groupr
 -21 -24 0 -25
 1050 5 0 4 3 3 7 1
 '94-pu-238'/
 300. 900. 2100.
 1.e10 1.e5 1.e4 1000. 100. 10. 1
 .1 0.025 0.8208e06 1.4e06
 3 1 'total'/
 3 2 'elastic'/
 3 16 'n2n'/
 3 17 'n3n'/
 3 18 'fission'/
 3 102 'capture'/
 3 251 'mubar'/
 3 252 'xi'/
 3 253 'gamma'/
 3 259 '1/v'/
 6 2 'elastic'/
 6 16 'n2n'/
 6 17 'n,3n'/
 6 18 'fission'/
 6 51 'discrete inelastic'/
 6 -59 'continued'/
 6 91 'continuum inelastic'/
 0/
 3 1 'total'/
 3 2 'elastic'/
 3 18 'fission'/
 3 102 'capture'/
 6 2 'elastic'/
 0/
 3 1 'total'/
 3 2 'elastic'/
 3 18 'fission'/
 3 102 'capture'/
 6 2 'elastic'/
 0/
 0/
 ccccr
 -25 26 27 0
 1 1 't2lanl njoy'/
 'ccccr tests for njoy'/
 50 0 1 4 1
 pu238 pu238 endfb4 ' 1050 ' 1050 10.89
 1 0 50 -1
 0 2.3821e02 3.3003e-11 1.7461e-12 0. 1.e10 0.0
 3 6
 300 900 2100
 1.e5 1.e4 1000. 100. 10. 1
 moder
 -24 28
 stop
EOF
../xnjoy<input
echo 'saving output and pendf files'
cp output out02
cp tape28 pend02

\end{ccode}
\normalsize

The \hyperlink{sRECONRhy}{RECONR} input is similar to the one
in Problem 1, but reading the listing file will demonstrate that
\hyperlink{sRECONRhy}{RECONR} added about 2700
points to the original grid in the resonance region.  After a
little thinning, it ended up with about 2800 resonance energy
points at 0K for 0.5\% reconstruction.  The data from
\hyperlink{sRECONRhy}{RECONR} was then passed to
\hyperlink{sBROADRhy}{BROADR} for preparation of cross sections
at 300, 900, and 2100K.  An examination of the output listing
shows that some thinning was possible because of the smoothing
effect of Doppler broadening; the zero degree grid of 3241 points
changed to 2341 points at 2100 degrees.
\index{RECONR}
\index{BROADR}

One of the new features of this run is the call to
\hyperlink{sUNRESRhy}{UNRESR}.\index{UNRESR}  The
user will normally notice that unresolved-resonance parameters
are available for a material by reading the introductory
information on the ENDF tape, but
\hyperlink{sUNRESRhy}{UNRESR} will return gracefully
if no unresolved data are present in the evaluation.  The long output
from \hyperlink{sUNRESRhy}{UNRESR} gives a table
of the self-shielded\index{self-shielding}
cross sections by reaction (vertical) and  sigma-zero\index{$\sigma_0$}
value (horizontal) for each point of the unresolved-range energy grid.
The reactions are flux-weighted total, elastic, fission, capture, and
current-weighted total reading from the top down.  Heating, damage,
and thermal cross sections were omitted for this particular problem.
Therefore, the final PENDF tape is on \cword{tape22}, the output
from \hyperlink{sUNRESRhy}{UNRESR}.

The \hyperlink{sGROUPRhy}{GROUPR}\index{GROUPR} run
shown here adds several new features over
the one given for Problem 1.  First, multiple temperatures and sigma-zero
values are specified in order to get tables of self-shielded
cross sections for the total, elastic, fission, and capture reactions.
It is desirable to use the same sigma-zero grid in
\hyperlink{sGROUPRhy}{GROUPR} that was used
in \hyperlink{sUNRESRhy}{UNRESR}, although
\hyperlink{sGROUPRhy}{GROUPR} will attempt to interpolate if the
grids are different.  Note that a section of
\hyperlink{sGROUPRhy}{GROUPR} input is given for 300K with
complete coverage of all the reactions, and additional sections are
given for the two higher temperatures with only the self-shielded
reactions included.\index{self-shielding}  An examination of the output
listing will show that self-shielded cross sections are given for the
low-energy reactions (total, elastic, fission, capture), with group
index reading down and sigma-zero value reading across.

Another new feature of this input is the computation of the
fission matrix (\cword{mfd=6}, \cword{mtd=18}).  More complicated
input may be necessary for other materials.  Examination of the
output listing will show that \hyperlink{sGROUPRhy}{GROUPR}
discovered that the fission spectrum shape was constant over
the entire energy range; therefore, it only put out a spectrum
and a fission-neutron production cross section.
\index{fission matrix}

When the \hyperlink{sGROUPRhy}{GROUPR} run is complete,
the final GENDF tape will be found on \cword{tape25}.  This file is
used as input to the \hyperlink{sCCCCRhy}{CCCCR}
module to produce ISOTXS and BRKOXS files for $^{238}$Pu.
\index{GENDF}
\index{CCCCR}
\index{ISOTXS}
\index{BRKOXS}


\subsection{Test Problem 3}
\label{ssMandT_3}
\index{testing!Problem 03}

This problem demonstrates the use of
\hyperlink{sGAMINRhy}{GAMINR}\index{GAMINR} to prepare
photon interaction (or photoatomic) cross sections.
\index{!photoatomic!photoatomic cross sections}
\index{photon interaction cross sections}
It also demonstrates \hyperlink{sDTFRhy}{DTFR}\index{DTFR}, including plotting,
\index{plotting} and \hyperlink{sMATXSRhy}{MATXSR}\index{MATXSR}.  For
the sake of continuity, this test problem uses a rather old version
of the photon interaction files called DLC7E\index{DLC7E}.  The MF23
and MF27 parts of these data are written on two separate files.  Later
libraries have everything on a single file.

\small
\begin{ccode}

echo 'NJOY Test Problem 3'
echo 'getting photoatomic tape gam23'
cp ../gam23 tape30
echo 'getting photoatomic tape gam27'
cp ../gam27 tape32
echo 'running njoy'
cat>input <<EOF
 reconr
 30 31
 'pendf tape for photon interaction cross sections from dlc7e'/
 1 1 0
 .001/
 '1-hydrogen'/
 92 1 0
 .001/
 '92-uranium'/
 0/
 gaminr
 32 31 0 33
 1 3 3 4 1
 '12 group photon interaction library'/
 -1 0/
 92
 -1 0/
 0/
 dtfr
 33 34 31 36
 1 1 0
 5 12 4 5 16 1 0
 'pheat'
 1 621 1
 0/
 'h' 1 1 0./
 'u' 92 1 0./
 /
 matxsr
 0 33 35/
 1 't2lanl njoy'/
 1 1 1 2
 '12-group photon interaction library'/
 'g'
 12
 'gscat'/
 1
 1
 'h' 1 1
 'u' 92 92
 viewr
 36 37/
 stop
EOF
../xnjoy<input
echo 'saving output and plot files'
cp output out03
cp tape37 plot03

\end{ccode}

\normalsize

This run starts with an application of the
\hyperlink{sRECONRhy}{RECONR}\index{RECONR}
module to linearize and unionize the File 23 cross sections.  Of course,
there is no resonance reconstruction required here.  This
\hyperlink{sRECONRhy}{RECONR} run demonstrates the use
of a material loop; both hydrogen and uranium are processed
in one run.  \hyperlink{sGAMINRhy}{GAMINR}\index{GAMINR} is then used
to prepare the multigroup photon interaction cross sections, including
heating; this run also uses a multimaterial loop.  The
LANL\index{Los Alamos National Laboratory!LANL} 12-group photon
structure\index{12-group structure} is used.  The GENDF tape
(\cword{tape33}) is processed into two completely different library
formats.  First, \hyperlink{sDTFRhy}{DTFR}\index{DTFR} is
called.  This is a rather old output module, but it is still useful
for some purposes.  For one thing,
it automatically produces plots\index{plotting} of the multigroup data
overlayed with the PENDF data.  Examples of the plots are given in the
\hyperlink{sDTFRhy}{DTFR} chapter of this manual.  The MATXS
output is more useful, because it can be used in a much more
flexible way by the TRANSX code.

\subsection{Test Problem 4}
\label{ssMandT_4}
\index{testing!Problem 04}

This problem illustrates several aspects of the calculation of
covariances (uncertainties) of multigroup data using
\hyperlink{sERRORRhy}{ERRORR}.\index{ERRORR} The first
\hyperlink{sERRORRhy}{ERRORR} problem produces, on unit 23,
a file of multigroup cross section covariances\index{covariances}
(\cword{mfcov=33}) for all reactions present (\cword{iread=0})
in File 33 for $^{235}$U.  The group structure employed is identical
to the energy grid selected by the evaluator (\cword{ign=19});
however, no covariances are produced for cross sections below 1 eV
or above 1 keV.  The second \hyperlink{sERRORRhy}{ERRORR}
run adds to the above results a second data set containing
multigroup $\overline{\nu}$ covariances.  Note that the
use of \cword{mfcov=31} dictates that a GENDF file be
produced (\cword{ngout=24}) prior to the start of
\hyperlink{sERRORRhy}{ERRORR}.  In the
$\overline{\nu}$ run, a user-defined group structure is employed
(\cword{ign=1}).

\small
\begin{ccode}

echo 'NJOY Test Problem 4'
echo 'getting endf tape 511'
cp ../t511 tape20
echo 'running njoy...'
cat>input <<EOF
 moder
 20 -21
 reconr
 -21 -22
 'u-235 10% pendf for errorr test problem from t511'/
 1395/
 .10/
 0/
 errorr
 -21 -22 0 23 0/
 1395 19 3 1 1/
 0 0./
 0 33/
 1
 1.e0 1.e3
 groupr
 -21 -22 0 24
 1395 3 0 3 0 1 1 1
 'u-235 multigroup nubar calculation'/
 0.
 1.e10
 3 452 'total nubar'/
 0/
 0/
 errorr
 -21 0 24 25 23/
 1395 1 2 1 1/
 1 0
 0 31/
 7
 1.e0 1.e1 1.e2 1.e3 1.e4 1.e5 1.e6 1.e7
 stop
EOF
../xnjoy<input
echo 'saving output'
cp output out04

\end{ccode}
\normalsize

\subsection{Test Problem 5}
\label{ssMandT_5}
\index{testing!Problem 05}

This short problem produces multigroup covariances for carbon,
again using \cword{ign=19}, but here all cross sections from
\cword{1E-5} to \cword{2E7} are treated.  The
\hyperlink{sCOVRhy}{COVR}\index{COVR} module
reads the binary output file from
\hyperlink{sERRORRhy}{ERRORR}\index{ERRORR} and produces
publication-quality plots\index{plotting} of all reactions for which
covariance data exist.  This simple problem setup could be used with
only a few simple changes for any nuclide, in order to take a quick
look at the contents of the covariance files.  For applications
where the resonance region is of interest, it is necessary to replace
the second \hyperlink{sMODERhy}{MODER}\index{MODER} step with a
resonance reconstruction step using
\hyperlink{sRECONRhy}{RECONR}, as in the previous example.

\small
\begin{ccode}

echo 'NJOY Test Problem 5'
echo 'getting endf tape 511'
cp ../t511 tape30
echo 'running njoy...'
cat>input <<EOF
 moder
 30 -31
 moder
 -31 -32
 errorr
 -31 -32 0 33/
 1306 19 2 1/
 0 0
 0 33/
 1
 1e-5 2e7/
 covr
 33 0 34/
 1/
 /
 /
 1306/
 viewr
 34 35/
 stop
EOF
../xnjoy<input
echo 'saving output and plot files'
cp output out05
cp tape35 plot05

\end{ccode}
\normalsize

\subsection{Test Problem 6}
\label{ssMandT_6}
\index{testing!Problem 06}

This test problem demonstrates and tests a number of different
kinds of plots\index{plotting} using data from ENDF/B-V Tape 511
(the ``Standards Tape'').  The graphs produced and a detailed
discussion of the input cards will be found in the
\hyperlink{sPLOTRhy}{PLOTR}\index{PLOTR}
chapter of this manual.

\small
\begin{ccode}

echo 'NJOY Test Problem 6'
echo 'getting endf tape 511'
cp ../t511 tape30
echo 'running njoy'
cat>input <<EOF
 plotr
 31/
 /
 1/
 '<endf/b-v carbon'/
 '<t>otal <c>ross <s>ection'/
 4/
 1e3 2e7/
 /
 .5 10/
 /
 5 30 1306 3 1/
 /
 1/
 '<endf/b-v carbon'/
 '(n,]a>) with fake data'/
 1 0 2 1 1.3e7 .32/
 /
 /
 /
 /
 5 30 1306 3 107/
 /
 '<endf/b-v mat1306'/
 2/
 0/
 -1 0/
 '<s>mith & <s>mith 1914'/
 0/
 1.1e7 .08 .05 .05/
 1.2e7 .10 .05 .05/
 1.3e7 .09 .04 .04/
 1.4e7 .08 .03 .03/
 /
 3/
 0/
 -1 2/
 '<b>lack & <b>lue 2008'/
 0/
 1.15e7 .07 .02 0. .2e6 0./
 1.25e7 .11 .02 0. .2e6 0./
 1.35e7 .08 .015 0. .2e6 0./
 1.45e7 .075 .01 0. .2e6 0./
 /
 1/
 '<endf/b-v carbon'/
 '<e>lastic <mf4>'/
 -1 2/
 /
 /
 /
 /
 /
 /
 5 30 1306 4 2/
 /
 1/
 '<endf/b-v l>i-6'/
 '(n,2n)]a >neutron distribution'/
 -1 2/
 /
 /
 0 12e6 2e6/
 /
 /
 /
 5 30 1303 5 24/
 /
 1/
 '<endf/b-v l>i-6'/
 '(n,2n)]a >neutron spectra vs <E>'/
 4 0 2 2/
 10. 2.e7/
 /
 1e-11 1e-6/
 '<c>ross <s>ection (barns/e<v>)'/
 5 30 1303 5 24 0. 12/
 /
 '10 <m>e<v'/
 1e3 2e-11 1e2/
 2/
 5 30 1303 5 24 0. 16/
 /
 '14 <m>e<v'/
 1e4 2e-10 2e3/
 3/
 5 30 1303 5 24 0. 20/
 /
 '20 <m>e<v'/
 1e5 2e-9 4e4/
 99/
 viewr
 31 32/
 stop
EOF
../xnjoy<input
echo 'saving plot file'
cp tape32 plot06

\end{ccode}
\normalsize

\subsection{Test Problem 7}
\label{ssMandT_7}
\index{testing!Problem 07}

This test problem demonstrates the preparation of
ACE format\index{ACE format} libraries for the MCNP\index{MCNP}
continuous-energy Monte Carlo\index{Monte Carlo} code.  The
material selected for processing was $^{235}$U from ENDF/B-V.
The \hyperlink{sGROUPRhy}{GROUPR} run is included to make a 30x20
photon production matrix for \hyperlink{sACERhy}{ACER}.  This is
an obsolete method for handling
photon production for MCNP -- nowadays, people use the
``Detailed Photon Production'' option.  This old test problem
is kept for consistency with the testing in previous versions
of NJOY.

\small
\begin{ccode}

echo 'NJOY Test Problem 7'
echo 'getting endf tape 511'
cp ../t511 tape20
echo 'running njoy...'
cat>input <<EOF
 moder
 20 -21
 reconr
 -21 -22
 'pendf tape for u-235 from endf/b-v tape 511' /
 1395 3 /
 .005 /
 '92-u-235 from endf/b-v tape 511 ' /
 'processed by the njoy nuclear data processing system' /
 'see original endf/b-v tape for details of evaluation' /
 0 /
 broadr
 -21 -22 -23
 1395 1 0 1 0 /
 .005 /
 300.
 0 /
 heatr
 -21 -23 -24/
 1395/
 moder
 -24 28
 groupr
 -21 -24 0 -25
 1395 3 2 9 0 1 1 1 /
 'u-235 from tape 511' /
 300.
 1.0e10
 16 /
 0 /
 0 /
 acer
 -21 -24 -25 26 27 /
 1/
 'njoy test problem 7'/
 1395 300. /
 0/
 /
 stop
EOF
../xnjoy<input
echo 'saving output, pendf, and ace files'
cp output out07
cp tape26 ace07
cp tape28 pend07

\end{ccode}
\normalsize

\subsection{Test Problem 8}
\label{ssMandT_8}
\index{testing!Problem 08}

This problem was added to the NJOY testing suite to verify the
processing of a typical ENDF/B-VI material with Reich-Moore
resonances and File 6 energy-angle distributions.  The $^{61}$Ni
evaluation is from ORNL, and the file is included in the NJOY
package.  The normal processing sequence is used.  The
\hyperlink{sGROUPRhy}{GROUPR} run is included to provide
additional output for comparing to standard results or to older
testing results.  An \hyperlink{sACERhy}{ACER} run is also included.

\small
\begin{ccode}

echo 'NJOY Test Problem 8'
echo 'getting endf tape for ni-61'
cp ../eni61 tape20
cat>input <<EOF
 moder
 20 -21
 reconr
 -21 -22
 'pendf tape for endf/b-vi.1 28-ni-61a'/
 2834 1 0 /
 .01/
 '28-ni-61a from endf/b-vi.1 t124 (hetrick,fu;ornl)'/
 0/
 broadr
 -21 -22 -23
 2834 1/
 .01/
 300/
 0/
 heatr
 -21 -23 -24/
 2834 6 0 1 0 2/
 302 303 304 402 443 444
 moder
 -24 28
 groupr
 -21 -24 0 -22
 2834 3 3 9 4 1 1 1
 'ni61a endf/b-vi.1 30x12'/
 300
 1e10
  3/
  3 251 'mubar'/
  3 252 'xi'/
  3 253 'gamma'/
  3 259 '1/v'/
  6/
  16/
 0/
 0/
 acer
 -21 -24 0 25 26
 1 1 1/
 '28-ni-61a from endf-vi.1'/
 2834 300./
 0/
 /
 stop
EOF
echo "running njoy"
../xnjoy<input
echo 'saving output and pendf files'
cp output out08
cp tape28 pend08

\end{ccode}
\normalsize

\subsection{Test Problem 9}
\label{ssMandT_9}
\index{testing!Problem 09}

This example demonstrates and tests the use of the
\hyperlink{sLEAPRhy}{LEAPR} module
to generate thermal scattering data.  The material is $^{1}$H
to be used with $^{16}$O for a problem requiring water.  The
run starts out with the familiar use of
\hyperlink{sRECONRhy}{RECONR} and
\hyperlink{sBROADRhy}{BROADR} to construct the basic cross sections
at 296K.  \hyperlink{sLEAPRhy}{LEAPR} is then
used to generate an $S(\alpha,\beta)$ scattering kernel on
\cword{tape24}.  Shortened $\alpha$ and $\beta$ grids are used
for compactness.  The \hyperlink{sTHERMRhy}{THERMR}
module is then run using \cword{tape24}
as input to add the thermal scattering cross section and
scattering data to the PENDF tape on \cword{tape23}.  The
new PENDF tape is on \cword{tape25}, and it can be used as
input to \hyperlink{sGROUPRhy}{GROUPR} for a multigroup library
or to \hyperlink{sACERhy}{ACER} for an
MCNP Monte Carlo library.

\small
\begin{ccode}

echo 'NJOY Test Problem 9'
echo 'getting endf tape 511'
cp ../t511 tape20
echo 'running njoy'
cat>input <<EOF
 moder
 20 -21
 reconr
 -21 -22
 'pendf tape for h-1 from endf/b tape 511'/
 1301 3/
 .005/
 '1-h-1 from tape 511'/
 'processed by the njoy nuclear data processing system'/
 'see original endf/b-v tape for details of evaluation'/
 0/
 broadr
 -21 -22 -23
 1301 1/
 .005/
 296.
 0/
 leapr
 24
 'h in h2o, shortened endf model'/
 1 1/
  101 1001./
  0.99917 20.449 2 0 0/
  1 1. 15.85316 3.8883 1/
 65 75 1/
  .01008 .015 .0252 .033 0.050406
  .0756 0.100812 0.151218 0.201624 0.252030 0.302436 0.352842
  0.403248 0.453654 0.504060 0.554466 0.609711 0.670259 0.736623
  0.809349 0.889061 0.976435 1.072130 1.177080 1.292110 1.418220
  1.556330 1.707750 1.873790 2.055660 2.255060 2.473520 2.712950
  2.975460 3.263080 3.578320 3.923900 4.302660 4.717700 5.172560
  5.671180 6.217580 6.816500 7.472890 8.192280 8.980730 9.844890
  10.79190 11.83030 12.96740 14.21450 15.58150 17.07960 18.72080
  20.52030 22.49220 24.65260 27.02160 29.61750 32.46250 35.58160
  38.99910 42.74530 46.85030 50.0/
  0.000000 0.006375 0.012750 0.025500 0.038250 0.051000 0.065750
  .0806495 0.120974 0.161299 0.241949 0.322598 0.403248 0.483897
  0.564547 0.645197 0.725846 0.806496 0.887145 0.967795 1.048440
  1.129090 1.209740 1.290390 1.371040 1.451690 1.532340 1.612990
  1.693640 1.774290 1.854940 1.935590 2.016240 2.096890 2.177540
  2.258190 2.338840 2.419490 2.500140 2.580790 2.669500 2.767090
  2.874450 2.992500 3.122350 3.265300 3.422470 3.595360 3.785490
  3.994670 4.224730 4.477870 4.756310 5.062580 5.399390 5.769970
  6.177660 6.626070 7.119240 7.661810 8.258620 8.915110 9.637220
  10.43200 11.30510 12.26680 13.32430 14.48670 15.76600 17.17330
  18.72180 20.42450 22.29760 24.35720 25.0/
 296/
  .00255 67/
    0 .0005 .001 .002 .0035 .005 .0075 .01 .013 .0165 .02 .0245
    .029 .034 .0395 .045 .0506 .0562 .0622 .0686 .075 .083 .091
    .099 .107 .115 .1197 .1214 .1218 .1195 .1125 .1065 .1005 .09542
    .09126 .0871 .0839 .0807 .07798 .07574 .0735 .07162 .06974
    .06804 .06652 .065 .0634 .0618 .06022 .05866 .0571 .05586
    .05462 .0535 .0525 .0515 .05042 .04934 .04822 .04706 .0459
    .04478 .04366 .04288 .04244 .042 0./
 0.055556 0. 0.444444 0./  for diffusion, use 120. for second number
  2/
  .205 .48/
  .166667 .333333/
 ' h(h2o) thermal scattering '/
 ' '/
 ' temperatures = 296 deg k. '/
 ' '/
 ' shortened version of the endf/b-vi.4 evaluation for '/
 ' hydrogen in water.  the energy transfer is limited to '/
 ' 0.625 ev.  this is only for njoy testing, not for '/
 ' real applications. '/
 /
 thermr
 24 -23 -25
 101 1301 8 1 2 1 0 1 222 2
 296.
 .05 .625
stop
EOF
../xnjoy<input
echo 'saving output and pendf files'
cp output out09
cp tape24 pend09

\end{ccode}
\normalsize

\subsection{Test Problem 10}
\label{ssMandT_10}
\index{testing!Problem 10}

This case was constructed to demonstrate unresolved resonance range
processing for MCNP libraries.  It starts with linearization,
unionization, resonance reconstruction, and Doppler broadening
to 300, 900, and 2100K.  The \hyperlink{sUNRESRhy}{UNRESR}
module to run to generate one version of the self-shielded
unresolved range cross sections on MF=2/MT152 to \cword{tape24}
for the three temperatures and seven values of the background
cross section $\sigma_0$.  Then the \hyperlink{sPURRhy}{PURR}
module is run to generate probability tables on MF=2/MT153 of the
PENDF tape and an alternate set of self-shielded cross
sections.  Actually, once \hyperlink{sPURRhy}{PURR} has run, the
MT152 values for \hyperlink{sUNRESRhy}{UNRESR} are overwritten
with the \hyperlink{sPURRhy}{PURR} values.  Both methods are
provided on this run to allow them to be compared.  Finally,
\hyperlink{sACERhy}{ACER} is run to format the data for MCNP.

\small
\begin{ccode}

echo 'NJOY Test Problem 10'
echo 'getting endf tape 404'
cp ../t404 tape20
echo 'running njoy'
cat>input <<EOF
 moder
 20 -21
 reconr
 -21 -22
 'pendf tape for pu-238 from endf/b-iv tape 404'/
 1050 3/
 .005/
 '94-pu-238 from endf/b tape t404'/
 'processed by the njoy nuclear data processing system'/
 'see original endf/b-iv tape for details of evaluation'/
 0/
 broadr
 -21 -22 -23
 1050 3 0 1 0/
 .005/
 300. 900. 2100.
 0/
 unresr
 -21 -23 -24
 1050 3 7 1
 300 900 2100
 1.e10 1.e5 1.e4 1000. 100. 10. 1
 0/
 purr
 -21 -24 -25
 1050 3 7 20 4/
 300 900 2100
 1.e10 1.e5 1.e4 1000. 100. 10. 1
 0/
 acer
 -21 -25 0 26 27/
 1/
 'njoy test problem 10'/
 1050 300./
 /
 /
 moder
 -25 28
 stop
EOF
../xnjoy<input
echo 'saving output, ace, and pendf files'
cp output out10
cp tape26 ace10
cp tape28 pend10

\end{ccode}
\normalsize

\subsection{Test Problem 11}
\label{ssMandT_11}
\index{testing!Problem 11}

This case demonstrates and tests the construction of a multigroup
library for the WIMS reactor physics code.  Here, we will need
thermal data, so the \hyperlink{sTHERMRhy}{THERMR} module
is included in the processing path.  Once the
\hyperlink{sGROUPRhy}{GROUPR} module has been run,
\hyperlink{sWIMSRhy}{WIMSR} is used for the final formatting.

\small
\begin{ccode}

echo 'NJOY Test Problem 11'
echo 'getting endf tape 404'
cp ../t404 tape20
echo 'running njoy'
cat>input <<EOF
 moder
 20 -21
 reconr
 -21 -22
 'pendf tape for pu-238 from endf/b-iv tape 404'/
 1050 3/
 .005/
 '94-pu-238 from endf/b tape t404'/
 'processed by the njoy nuclear data processing system'/
 'see original endf/b-iv tape for details of evaluation'/
 0/
 broadr
 -21 -22 -23
 1050 3 0 1 0 /
 .005/
 300. 900. 2100.
 0/
 unresr
 -21 -23 -24
 1050 3 7 1
 300 900 2100
 1.e10 1.e5 1.e4 1000. 100. 10. 1
 0/
 thermr
 0 -24 -25
 0 1050 8 3 1 0 0 1 221 0
 300. 900. 2100.
 .05 4.2
 groupr
 -21 -25 0 -26
 1050 9 0 5 3 3 7 1
 '94-pu-238'/
 300. 900. 2100.
 1.e10 1.e5 1.e4 1000. 100. 10. 1
 3 1 'total'/
 3 2 'elastic'/
 3 16 'n2n'/
 3 17 'n3n'/
 3 18 'fission'/
 3 102 'capture'/
 3 221 'free gas thermal'/
 6 2 'elastic'/
 6 16 'n2n'/
 6 17 'n,3n'/
 6 18 'fission'/
 6 51 'discrete inelastic'/
 6 -59 'continued'/
 6 91 'continuum inelastic'/
 6 221 'free gas thermal'/
 0/
 3 1 'total'/
 3 2 'elastic'/
 3 18 'fission'/
 3 102 'capture'/
 3 221 'free gas thermal'/
 6 2 'elastic'/
 6 221 'free gas thermal'/
 0/
 3 1 'total'/
 3 2 'elastic'/
 3 18 'fission'/
 3 102 'capture'/
 3 221 'free gas thermal'/
 6 2 'elastic'/
 6 221 'free gas thermal'/
 0/
 0/
 wimsr
 -26 27
 1/
 1050 1 1050./
 3 7 1e10 3 10.890 221 0/
 1. 1. 1. 1. 1. 1. 1. 1. 1. 1. 1. 1. 1./
 stop
EOF
../xnjoy<input
echo 'saving output and wims files'
cp output out11
cp tape27 wims11

\end{ccode}
\normalsize

\subsection{Test Problem 12}
\label{ssMandT_12}
\index{testing!Problem 12}

This case was constructed to demonstrate the methods for adding
gas production to a processing stream.  The
\hyperlink{sGASPRhy}{GASPR} module doesn't
require any special input.  We then use the
\hyperlink{sPLOTRhy}{PLOTR} and \hyperlink{sVIEWRhy}{VIEWR}
modules to graph the gas production reactions using color
Postscript.  The gas production reactions have special MT
values in the range 203 to 207.  They can also be fed into
GROUPR for multigroup applications and
\hyperlink{sACERhy}{ACER} for MCNP Monte Carlo libraries.

\small
\begin{ccode}

echo 'NJOY Test Problem 12'
echo 'getting endf tape for ni-61'
cp ../eni61 tape20
cat>input <<EOF
 reconr
 20 21
 'pendf tape for endf/b-vi.1 28-ni-61a'/
 2834 1 0 /
 .01/
 '28-ni-61a from endf/b-vi.1 t124 (hetrick,fu;ornl)'/
 0/
 gaspr
 20 21 22
 plotr
 23/
 1 1 .3 2/
 1 3/
 '<endf/b-vi n>i-61'/
 '<r>esonance <c>ross <s>ections'/
 2 0 3 1 23e3 5e2/
 .5e4 3e4 .5e4/
 /
 1e-3 1e3/
 /
 6 22 2834 3 2/
 0 0 0 3 2/
 'elastic'/
 2/
 6 22 2834 3 102/
 0 0 0 1 2/
 'capture'/
 1 7/
 '<endf/b-vi n>i-61'/
 '<g>as <p>roduction'/
 1 0 3 1/
 0 2e7 5e6/
 /
 /
 /
 6 22 2834 3 203 0./
 0 0 0 1 2/
 'hydrogen'/
 2/
 6 22 2834 3 207 0./
 0 0 0 2 2/
 'helium-4'/
 99/
 viewr
 23 24
 stop
EOF
echo "running njoy"
../xnjoy<input
echo 'saving output, pendf, and plot files'
cp output out12
cp tape22 pend12
cp tape24 plot12

\end{ccode}
\normalsize

\subsection{Test Problem 13}
\label{ssMandT_13}
\index{testing!Problem 13}

This test problem was constructed when Reich-Moore resonance
processing and File 6 energy-angle distributions began to be
used in ENDF/B files.  The sample evaluation for $^{61}$Ni was
generated at ORNL, and it is provided in the NJOY package.
Note that \hyperlink{sHEATRhy}{HEATR} and
\hyperlink{sGASPRhy}{GASPR} are included to satisfy the
expectations for a complete ACE library.  Two
\hyperlink{sACERhy}{ACER} runs are used here.  The first one
generates an ACE library on \cword{tape26}.  The second one reads
in this new ACE file, does a series of consistency checks, provides
an eye-readable listing, and makes a large set of color Postscript
plots showing the various reaction cross sections and
distributions.  This is a part of the QA procedure for ACE
libraries.  The file \cword{tape36} output by
\hyperlink{sVIEWRhy}{VIEWR} is the plot file in Postscript
format.

\small
\begin{ccode}

echo 'NJOY Test Problem 13'
echo 'getting endf tape for ni-61'
cp ../eni61 tape20
cat>input <<EOF
 moder
 20 -21
 reconr
 -21 -22
 'pendf tape for endf/b-vi.1 28-ni-61a'/
 2834 1 0 /
 .01/
 '28-ni-61a from endf/b-vi.1 t124 (hetrick,fu;ornl)'/
 0/
 broadr
 -21 -22 -23
 2834 1/
 .01/
 300/
 0/
 heatr
 -21 -23 -24/
 2834 6 0 1 0 2/
 302 303 304 402 443 444
 gaspr
 -21 -24 -25
 moder
 -25 28
 acer
 -21 -25 0 26 27
 1 0 1/
 '28-ni-61a endf-vi.1 njoy99'/
 2834 300./
 /
 /
 acer
 0 26 33 34 35
 7 1 2/
 '28-ni-61a endf-vi.1 njoy99'/
 viewr
 33 36/
 stop
EOF
echo "running njoy"
../xnjoy<input
echo 'saving output, pendf, ace, and plot files'
cp output out13
cp tape28 pend13
cp tape26 ace13
cp tape36 plot13

\end{ccode}
\normalsize

\subsection{Test Problem 14}
\label{ssMandT_14}
\index{testing!Problem 14}

This case demonstrates and tests processing for incident charged
particles, specifically, protons on $^{14}$N.  It is not
necessary to run reconstruction or Doppler broadening, so
the evaluation is read directly into
\hyperlink{sACERhy}{ACER}.  A copy of the
ENDF/B file is used as the PENDF input.  Once again, two
\hyperlink{sACERhy}{ACER} runs are used: one to generate the ACE
file, and one to do the checks and make color Postscript plots.

\small
\begin{ccode}

echo 'NJOY Test Problem 14'
echo 'getting endf tape for p+n14'
cp ../epn14 tape20
cp tape20 tape21
cat>input <<EOF
 acer
 20 21 0 31 32
 1 0 1/
 'proton + 7-n-14 apt la150 njoy99 mcnpx'/
 725 0./
 /
 /
 acer
 0 31 33 34 35
 7 1 2/
 'proton + 7-n-14 apt la150 njoy99 mcnpx'/
 viewr
 33 36/
 stop
EOF
echo "running njoy"
../xnjoy<input
echo 'saving output, ace, and plot files'
cp output out14
cp tape31 ace14
cp tape36 plot14

\end{ccode}
\normalsize

\subsection{Test Problem 15}
\label{ssMandT_15}
\index{testing!Problem 15}

This test problem exercises the \hyperlink{sERRORRhy}{ERRORR}
routines used to process MF31 ($\bar{\nu}$), MF33 (cross sections)
and MF34 (MF4/MT2 P$_1$ component only) uncertainty
data.  We use the JENDL-3.3 $^{238}$U evaluation
for input, demonstrating that non-ENDF/B libraries that conform to
the ENDF-6 format can be processed by NJOY.  The job includes a
standard \hyperlink{sMODERhy}{MODER}/
\hyperlink{sRECONRhy}{RECONR}/
\hyperlink{sBROADRhy}{BROADR}/
\hyperlink{sGROUPRhy}{GROUPR} sequence to create Doppler
broadened PENDF and group-averaged GENDF tapes followed by a
sequence of \hyperlink{sERRORRhy}{ERRORR} jobs; one for each
MF value to be processed.  Use of \hyperlink{sGROUPRhy}{GROUPR}
to process $\bar{\nu}$ (MF=3, MT=whatever MF=31 sections are
present) is a pre-requisite for MF=31 processing with
\hyperlink{sERRORRhy}{ERRORR}.  The first
\hyperlink{sERRORRhy}{ERRORR} job processes MF=31 data
(specified via the second item on the third input line
(corresponding to card 7 in the
\hyperlink{sERRORRhy}{ERRORR} input
description)).  The first input line specifies the various input and
output files; \cword{tape21} for a binary ENDF tape, \cword{tape91}
for a GENDF tape and \cword{tape25} for the
\hyperlink{sERRORRhy}{ERRORR} output tape in
this instance.  The material number, multigroup energy structure and
weighting function options are specified on the second input line.
This input line could have been terminated here, as the fourth and
fifth items, related to printing and selection of relative rather than
absolute covariances, are given their default values.  Users will
notice that the group structure and weighting function options differ
from those specified in the
\hyperlink{sGROUPRhy}{GROUPR} input preceding
\hyperlink{sERRORRhy}{ERRORR}.  NJOY will
not flag this as an error, but it is not recommended practice.  The
third input card (corresponding to card 7 in the
\hyperlink{sERRORRhy}{ERRORR} input, since
\cword{ngout.ne.0}, and we are dealing with ENDF-6 format data),
specifies \cword{iread=0} meaning that all MF=31 (the second item on
this input card) MT values found on the ENDF input tape will be
processed.  The remaining items on this card are not germaine to
MF=31 processing and could have been omitted.  Additional
\hyperlink{sERRORRhy}{ERRORR} jobs process the cross section
uncertainties (MF=33) and elastic scattering Legendre (P$_1$ only)
moment uncertainty (MF=34) using a combination of ENDF and GENDF
input tapes.  Separate output files (\cword{tape25}, \cword{tape26}
and \cword{tape27}) are created with these three
\hyperlink{sERRORRhy}{ERRORR} executions.  These are ASCII files
that may be viewed in any text editor or saved for use in a
subsequent NJOY job, such as
\hyperlink{sCOVRhy}{COVR}/\hyperlink{sVIEWRhy}{VIEWR} for
plot generation.

\small
\begin{ccode}

echo 'NJOY Test Problem 15'
echo 'getting JENDL-3.3 U-238'
cp ../J33U238 tape20
echo 'running njoy'
ulimit -s 32768
cat>input <<EOF
moder
 20 -21 /
reconr
 -21 -22 /
 'processing of U-238 of jendl-3.3.'
9237 0 0 /
0.001 /
0 /
broadr
 -21 -22 -23 /
9237 1 0 0 0 /
0.001 /
300. /
0 /
groupr
 -21 -23 0 91 /
9237 3 0 2 1 1 1 1 /
'test'
300. /
1.0e10 /
3 /
3 251 'mubar' /
3 252 'xi' /
3 452 'nu' /
3 455 'nu' /
3 456 'nu' /
5 18  'xi' /
0 /
0 /
--
-- process mf31
errorr
 -21 0 91 25 0 0 /
9237 3 6 1 1 /
1 300./
0 31 1 1 -1 /
--
-- process mf33
errorr
 -21 0 91 26 0 0 /
9237 3 6 1 1 /
1 300./
0 33 1 1 -1 /
--
-- process mf34
errorr
 -21 0 91 27 0 0 /
9237 3 6 1 1 /
1 300./
0 34 1 1 -1 /
--
-- make mf31 plot file
covr
 25 0 35 /
 1 /
 /
 /
 9237 /
--
-- make mf33 plot file.
covr
 26 0 36 /
 1 /
 /
 1 14 /
 9237   1   9237   1 /
 9237   1   9237   2 /
 9237   2   9237   2 /
 9237   2   9237   4 /
 9237   2   9237  16 /
 9237   2   9237  17 /
 9237   2   9237  18 /
 9237   2   9237 102 /
 9237   4   9237   4 /
 9237  16   9237  16 /
 9237  17   9237  17 /
 9237  18   9237  18 /
 9237  18   9237 102 /
 9237 102   9237 102 /
--
-- make mf34 plot file.
covr
 27 0 37 /
 1 /
 /
 /
 9237 /
--
-- make mf31 postscript file.
viewr
 35 45 /
--
-- make mf33 postscript file.
viewr
 36 46 /
--
-- make mf34 postscript file.
viewr
 37 47 /
stop
EOF
../xnjoy<input
echo 'saving output file'
cp output out15
cp tape45 plot15-31
cp tape46 plot15-33
cp tape47 plot15-34

\end{ccode}
\normalsize

\subsection{Test Problem 16}
\label{ssMandT_16}
\index{testing!Problem 16}

This test problem is similar to test 15, but we omit
\hyperlink{sGROUPRhy}{GROUPR} and only
process cross section (MF=33) and P$_1$ moment (MF=34) uncertainties
with \hyperlink{sERRORRhy}{ERRORR}.  As stated in the previous
problem, \hyperlink{sGROUPRhy}{GROUPR} execution is
required when using \hyperlink{sERRORRhy}{ERRORR} to process MF=31
data but, as shown here, is not required to process MF=33 or
MF=34 uncertainty data.  We also append \hyperlink{sCOVRhy}{COVR}
and \hyperlink{sVIEWRhy}{VIEWR} to this job to demonstrate Postscript plot
generation.  The first \hyperlink{sCOVRhy}{COVR} job reads
\cword{tape26}, the \hyperlink{sERRORRhy}{ERRORR} MF=33
processed output tape and will create \cword{tape36} containing the
appropriate plot commands that \hyperlink{sVIEWRhy}{VIEWR}
uses to create a Postscript formatted \cword{tape46}.  Additional
\hyperlink{sCOVRhy}{COVR} input includes \cword{icolor=1}
(card 2) to specify color plots, and a user-specified number of plots,
\cword{ncase=14}, on the third (card 3a) input card.  This is followed
by 14 occurrences of card 4.  The second \hyperlink{sCOVRhy}{COVR}
job reads the \hyperlink{sERRORRhy}{ERRORR} output tape created
from MF=34 processing and creates all possible
plots as we use the default input values for the variables defined by
the third input card.  In reality there will only be a single plot
since MF=34 processing is only performed for the elastic scattering
(MF=4/MT=2) P$_1$ Legendre moment.  The output of this
\hyperlink{sCOVRhy}{COVR} job, \cword{tape37} is fed to
\hyperlink{sVIEWRhy}{VIEWR}, which generates a Postscript formatted
file, \cword{tape47}.

\small
\begin{ccode}

echo 'NJOY Test Problem 16'
echo 'getting JENDL-3.3 U-238'
cp ../J33U238 tape20
echo 'running njoy'
ulimit -s 32768
cat>input <<EOF
moder
20 -21 /
reconr
-21 -22 /
'processing of U-238 of jendl-3.3.'
9237 0 0 /
0.001 /
0 /
broadr
-21 -22 -23 /
9237 1 0 0 0 /
0.001 /
300. /
0 /
--
-- process mf33 with internal group averaging
-- since there is no groupr input tape
errorr
-21 -23 0 26 0 0 /
9237 3 6 1 1 /
0 300. /
0 33 1 /
--
-- process mf34
errorr
-21 -23 0 27 0 0 /
9237 3 6 1 1 /
 0 300. /
0 34 1 1 /
--
-- make mf33 plot file.
covr
 26 0 36 /
 1 /
 /
 1 14 /
 9237   1   9237   1 /
 9237   1   9237   2 /
 9237   2   9237   2 /
 9237   2   9237   4 /
 9237   2   9237  16 /
 9237   2   9237  17 /
 9237   2   9237  18 /
 9237   2   9237 102 /
 9237   4   9237   4 /
 9237  16   9237  16 /
 9237  17   9237  17 /
 9237  18   9237  18 /
 9237  18   9237 102 /
 9237 102   9237 102 /
--
-- make mf34 plot file.
covr
 27 0 37 /
 1 /
 /
 /
 9237 /
--
-- make mf33 postscript file.
viewr
 36 46 /
--
-- make mf34 postscript file.
viewr
 37 47 /
stop
EOF
../xnjoy<input
echo 'saving output file'
cp output out16
cp tape46 plot16-33
cp tape47 plot16-34

\end{ccode}
\normalsize

\subsection{Test Problem 17}
\label{ssMandT_17}
\index{testing!Problem 17}

Test problem 17 is the longest running job in the NJOY test suite,
involving processing of $^{235,238}$U and $^{239}$Pu.  We start with
a standard sequence of \hyperlink{sRECONRhy}{RECONR}/
\hyperlink{sBROADRhy}{BROADR}/
\hyperlink{sGROUPRhy}{GROUPR} for each of these nuclides.
Three separate GENDF files are created (\cword{tape91}, \cword{tape92},
and \cword{tape93}) by these \hyperlink{sGROUPRhy}{GROUPR}
executions, followed by a \hyperlink{sMODERhy}{MODER}
job that merges the three GENDF files onto a single GENDF
tape.  \hyperlink{sMODERhy}{MODER}'s
card 1 \cword{nin=2} option is specified, followed by multiple card 3
input to accomplish this merge.  \hyperlink{sERRORRhy}{ERRORR}
begins with the standard input
and output tape specifications on card 1 and the material number,
multigroup and weighting function specification on card 2.  As we've
included an GENDF tape as part of the input and we're dealing with
ENDF-6 formatted data, the third input card corresponds to card 7 in
the \hyperlink{sERRORRhy}{ERRORR} input description.  Here we
specify \cword{iread=2}, indicating that additional MAT/MT data are
to be included over those found on the original ENDF input
(\cword{tape21}).  This additional input follows on the fourth
and fifth input cards, corresponding to repeated use of card 10
from the \hyperlink{sERRORRhy}{ERRORR} input description.  Based upon
this additional input, the effect of $^{235}$U (\cword{mat}=9228) and
$^{239}$Pu (\cword{mat}=9437) MT=18 uncertainties are included when processing
$^{238}$U (\cword{mat}=9237) uncertainties.

\small
\begin{ccode}

echo 'NJOY Test Problem 17'
echo 'getting JENDL-3.3 U-238'
cp ../J33U238 tape21
echo 'getting JENDL-3.3 U-235'
cp ../J33U235 tape22
echo 'getting JENDL-3.3 Pu-239'
cp ../J33Pu239 tape23
echo 'running njoy'
ulimit -s 32768
cat>input <<EOF
 reconr
 21 41 /
 'processing jendl-3.3 238u.'/
 9237 0 0 /
 0.001 /
 0 /
 broadr
 21 41 31 /
 9237 1 0 0 0 /
 0.001 /
 300. /
 0 /
 reconr
 22 42 /
 'processing jendl-3.3 235u.'/
 9228 0 0 /
 0.001 /
 0 /
 broadr
 22 42 32 /
 9228 1 0 0 0 /
 0.001 /
 300. /
 0 /
 reconr
 23 43 /
 'processing jendl-3.3 239pu.'/
 9437 0 0 /
 0.001 /
 0 /
 broadr
 23 43 33 /
 9437 1 0 0 0 /
 0.001 /
 300. /
 0 /
 groupr
 21 31 0 91 /
 9237 3 0 6 1 1 1 0 /
 'u-238' /
 300. /
 1.0e10 /
 3 /
 3 251 'mubar' /
 3 252 'xi' /
 3 452 'nu' /
 3 455 'nu' /
 3 456 'nu' /
 5 18  'xi' /
 0 /
 0 /
 groupr
 22 32 0 92 /
 9228 3 0 6 1 1 1 0 /
 'u-235' /
 300. /
 1.0e10 /
 3 /
 0 /
 0 /
 groupr
 23 33 0 93 /
 9437 3 0 6 1 1 1 0 /
 'pu-239' /
 300. /
 1.0e10 /
 3 /
 0 /
 0 /
 moder
 2 99 /
 'merge u235, u-238 and pu-239' /
 92  9228 /
 91  9237 /
 93  9437 /
 0 /
 errorr
 21 0 99 26 0 /
 9237 3 6 1 /
 1 300./
 2 33 1 1 -1 /
 9228 18 /
 9437 18 /
 0 /
stop
EOF
../xnjoy<input
echo 'saving output file'
cp output out17

\end{ccode}
\normalsize

\subsection{Test Problem 18}
\label{ssMandT_18}
\index{testing!Problem 18}

This test problem exercises the \hyperlink{sERRORRhy}{ERRORR}
spectrum uncertainty processing routines followed by a
\hyperlink{sCOVRhy}{COVR}/\hyperlink{sVIEWRhy}{VIEWR} job to
produce plots of these data.  There are no spectrum uncertainty
data in the ENDF/B-VII.0 neutron transport files, but such data
are provided for the spontaneous fission of $^{252}$Cf in the
ENDF/B-VII.0 decay file.  Therefore a fictitious transport file has
been created containing MF5 and MF35, MT18 data from the
decay file with the remaining MF/MT data coming from the transport
file.  This tape serves as the ENDF input to a
\hyperlink{sMODERhy}{MODER}/\hyperlink{sRECONRhy}{RECONR}/
\hyperlink{sBROADRhy}{BROADR}/\hyperlink{sGROUPRhy}{GROUPR}/
\hyperlink{sERRORRhy}{ERRORR}/\hyperlink{sCOVRhy}{COVR}/
\hyperlink{sVIEWRhy}{VIEWR} job.

This job could have started at \hyperlink{sGROUPRhy}{GROUPR}
with MF5/MT18 processing.  The
multgroup user input mimics the energy structure appearing in the
MF35/MT18 input tape and was chosen to allow easy comparison of the
final calculated uncertainty with the corresponding data provided in
MF1/MT451 by the evaluator.  The \hyperlink{sGROUPRhy}{GROUPR}
GENDF output file resides on \cword{tape91} which along with the
original ENDF input (\cword{tape20}) are input to
\hyperlink{sERRORRhy}{ERRORR}.  The \hyperlink{sERRORRhy}{ERRORR}
output will go on \cword{tape28}.  This tape will contain the results
of processing MATN 9999 for the user-specified multigroup structure
with no weighting function (based upon the \cword{ign=1},
\cword{iwt=2} options specified on card 2).  The third line in the
\hyperlink{sERRORRhy}{ERRORR} input (really card 7 in the
ERRORR input description) specifies processing MF35 data for an
incident neutron energy of 1.23 MeV (based upon the \cword{-1, 1.23e6}
values appearing as the fourth and fifth items on this input card).
The remaining input identifies the user-specified multigroup structure.
After \hyperlink{sERRORRhy}{ERRORR}, we pass \cword{tape28} to
\hyperlink{sCOVRhy}{COVR}, where the appropriate
plot commands are generated to produce the correlation matrix plot,
the spectrum plot, and the uncertainty plot.  Card 2 (\cword{icolor})
requests a color contour plot, and card 3 sets the lower (x-axis)
energy limit to 1 keV with default values used for the other input
parameters.  Card 5 specifies the \cword{mat} number for the material of
interest.  Finally, \hyperlink{sVIEWRhy}{VIEWR} reads the
\hyperlink{sCOVRhy}{COVR} output file, \cword{tape38}, and produces
a postscript-formatted file, \cword{tape39}.

\small
\begin{ccode}

echo 'NJOY Test Problem 18'
echo 'getting endf tape for e7 cf252'
cp ../DCf252 tape20
echo 'running njoy'
cat>input <<EOF
--
-- Copy ascii input to binary.
moder
 20 -21 /
--
-- Resonance reconstruction, to 0.1%.
reconr
 -21 -22 /
 'processing e70 252Cf with decay mf5/mt18 & mf35/mt18'
 9999 0 0 /
 0.001 /
 0 /
--
-- Doppler broaden to 300K.
broadr
 -21 -22 -23 /
 9999 1 0 0 0 /
 0.001 /
 300. /
 0 /
--
-- Group average, 300K with mf35 group structure.
--  - All file 3 cross sections plus fission spectrum.
groupr
 -21 -23 0 91 /
 9999 1 0 2 1 1 1 1 /
 'test'
 300. /
 1.0e10 /
 71 / # of groups, energy boundaries follow:
 1.000000-5 1.500000+4 3.500000+4 5.500000+4 7.500000+4
 9.500000+4 1.150000+5 1.350000+5 1.650000+5 1.950000+5
 2.250000+5 2.550000+5 3.050000+5 3.550000+5 4.050000+5
 4.550000+5 5.050000+5 5.550000+5 6.050000+5 6.550000+5
 7.050000+5 7.550000+5 8.050000+5 8.550000+5 9.050000+5
 9.550000+5 1.050000+6 1.150000+6 1.250000+6 1.350000+6
 1.450000+6 1.550000+6 1.650000+6 1.750000+6 1.850000+6
 1.950000+6 2.150000+6 2.350000+6 2.550000+6 2.750000+6
 2.950000+6 3.250000+6 3.550000+6 3.850000+6 4.150000+6
 4.450000+6 4.750000+6 5.050000+6 5.550000+6 6.050000+6
 6.550000+6 7.050000+6 7.550000+6 8.050000+6 8.550000+6
 9.050000+6 9.550000+6 1.005000+7 1.055000+7 1.105000+7
 1.155000+7 1.205000+7 1.255000+7 1.305000+7 1.355000+7
 1.405000+7 1.460000+7 1.590000+7 1.690000+7 1.790000+7
 1.910000+7 2.000000+7
 3 /
 5 18  'chi' /
 0 /
 0 /
--
-- ERRORJ, mf35
errorr
 20 0 91 28 0 0 /
 9999 1 2 1 1 /
 1 300./
 0 35 1 1 -1 1.23e6 /
 71 / # of groups, energy boundaries follow:
 1.000000-5 1.500000+4 3.500000+4 5.500000+4 7.500000+4
 9.500000+4 1.150000+5 1.350000+5 1.650000+5 1.950000+5
 2.250000+5 2.550000+5 3.050000+5 3.550000+5 4.050000+5
 4.550000+5 5.050000+5 5.550000+5 6.050000+5 6.550000+5
 7.050000+5 7.550000+5 8.050000+5 8.550000+5 9.050000+5
 9.550000+5 1.050000+6 1.150000+6 1.250000+6 1.350000+6
 1.450000+6 1.550000+6 1.650000+6 1.750000+6 1.850000+6
 1.950000+6 2.150000+6 2.350000+6 2.550000+6 2.750000+6
 2.950000+6 3.250000+6 3.550000+6 3.850000+6 4.150000+6
 4.450000+6 4.750000+6 5.050000+6 5.550000+6 6.050000+6
 6.550000+6 7.050000+6 7.550000+6 8.050000+6 8.550000+6
 9.050000+6 9.550000+6 1.005000+7 1.055000+7 1.105000+7
 1.155000+7 1.205000+7 1.255000+7 1.305000+7 1.355000+7
 1.405000+7 1.460000+7 1.590000+7 1.690000+7 1.790000+7
 1.910000+7 2.000000+7
--
-- make plot file.
covr
 28 0 38 /
 1 /
 1.e3 /
 /
 9999 /
--
-- make postscript file.
viewr
 38 39 /
stop
EOF
../xnjoy<input
echo 'saving output, ace, and pendf files'
cp output out18
cp tape39 plot18

\end{ccode}
\normalsize

\subsection{Test Problem 19}
\label{ssMandT_19}
\index{testing!Problem 19}

The input file for this test is an evaluation for $^{241}$Pu using
Reich-Moore\index{Reich-Moore!RM} resonance parameters.

\small
\begin{ccode}

echo 'NJOY Test Problem 19'
echo 'getting endf tape for e6 pu241c'
cp ../e6pu241c tape20
echo 'running njoy'
cat>input <<EOF
 moder
 20 -21
 reconr
 -21 -22
 'pendf tape for pu-241 from endf/b-vi.3'/
 9443 3/
 .02/
 '94-pu-241 from endf/b-vi.3'/
 'processed by the njoy nuclear data processing system'/
 'see original endf/b-iv tape for details of evaluation'/
 0/
 broadr
 -21 -22 -23
 9443 3 0 1 0/
 .02/
 293.6 900. 2100.
 0/
 unresr
 -21 -23 -24
 9443 3 7 1
 293.6 900 2100
 1.e10 1.e5 1.e4 1000. 100. 10. 1
 0/
 heatr
 -21 -24 -25/
 9443 3/
 302 318 402
 purr
 -21 -25 -26
 9443 3 7 20 4/
 293.6 900 2100
 1.e10 1.e5 1.e4 1000. 100. 10. 1
 0/
 acer
 -21 -26 0 27 28/
 1/
 'njoy test problem 19'/
 9443 293.6/
 /
 /
 moder
 -26 29
 stop
EOF
../xnjoy<input
echo 'saving output, ace, and pendf files'
cp output out19
cp tape27 ace19
cp tape29 pend19

\end{ccode}
\normalsize

It starts with a normal \hyperlink{sRECONRhy}{RECONR} run
using 2\% reconstruction for economy.  The output file will
disclose that Reich-Moore resonances were processed.  The
rest of the run follows the pattern of the previous examples.

\subsection{Test Problem 20}
\label{ssMandT_20}
\index{testing!Problem 20}

The input file for this case is an experimental evaluation
from ORNL\index{Oak Ridge National Laboratory!ORNL} for $^{35}$Cl
that uses the Reich-Moore-Limited\index{Reich-Moore-Limited!RML}
(RML) format for its resonance parameters (\cword{lrf}=7 in
MF2/MT151) and for the covariances between resonance
parameters (MF32).  Utilizing the multi-channel capabilities
of the RML representation, resonance cross sections and
their covariances also include the first two inelastic levels
(MT=51 and 52) in addition to the normal elastic and capture
channels.

\small
\begin{ccode}

echo 'NJOY Test Problem 20'
echo 'getting cl35rml'
cp ../cl35rml tape20
echo 'running njoy'
ulimit -s 32768
cat>input <<EOF
errorr
 999     /  option to insert dummy file 33 data
 +20 +21 /  input & output tapes
   1     /  mt to insert,
   2     /  continue mt list ...
 102     /
 600     /
   0     /  terminate mt list with zero
reconr
 +21 +22 /
 'processing cl35 with rml'
 1725 0 0 /
 0.01 /
 0 /
broadr
 +20 +22 +23 /
 1725 1 0 0 0 /
 0.01 /
 293.6 /
 0 /
groupr
 +20 +23 0 +91 /
 1725 4 0 2 1 1 1 1 /
 'test of cl35 with rml'
 293.6 /
 1.0e10 /
 3 /
 3 251 'mubar' /
 3 252 'xi' /
 0 /
 0 /
errorr
 +21 0 +91 +25 0 0 /
 1725 4 2 1 1 /
 1 293.6/
 0 33 1 1 -1 /
covr
 +25 0 +26 /
 1 /
 .01 /
 1 /
 1725 /
viewr
 +26 +27
stop
EOF
../xnjoy<input
echo 'saving output file'
cp output out20
cp tape23 pend20
cp tape27 plot20

\end{ccode}
\normalsize

The reconr output from this run will show that it used the RML
methods and it will list the reactions treated with resonance
methods (MT=2, 51, 52, 102).  \hyperlink{sBROADRhy}{BROADR}
and \hyperlink{sGROUPRhy}{GROUPR} are used in the normal way,
\hyperlink{sERRORRhy}{ERRORR} and \hyperlink{sCOVRhy}{COVR}
are then run to produce the covariances and prepare plots of the
results.


\subsection{Application of the NJOY System}
\label{ssMandT_application}

There are not enough standard test problems to exercise all the many
options that evaluators manage to use.  Therefore, the next step
in testing is to run NJOY on many different evaluations.  This
has included running all the materials in recent ENDF/B (as well as
many JEFF, JENDL and special purpose libraries such as the IAEA's
International Reactor Dosimetry and Fusion File, IRDFF) neutron,
thermal, photoatomic, photonuclear, and charged-particle
sublibraries through to MCNP ``ACE" files followed by
short MCNP runs to make sure the data have no obvious flaws
that prevent the code from running.  The output files were examined
for error messages and other indications of problems.
This process also produces a useful set of plots that can be
scanned for possible problems.  Plots from some of these
tests are available on the web.  Browse to
\href{http://t2.lanl.gov}{http://t2.lanl.gov} and select the
``Nuclear Information Service (NIS)", ``Data area"
and ``US ENDF/B Libraries" links.
Look in ``ENDF/B-VII Neutron Data,'' ``ENDF/B-VII Thermal
Scattering Data,'' ``ENDF/B-VII Proton Data,'' and ``ENDF/B-VII
Photonuclear Data.

As discussed in the \hyperlink{sHEATRhy}{HEATR} section of this
report, NJOY has the capability to examine ENDF-format evaluations
for energy balance problems that show up for nuclear heating
calculations.  These tests can also reveal problems with the
processing on occasion.  A large set of results, including plots,
can be found on the web.  As before, browse to
\href{http://t2.lanl.gov}{http://t2.lanl.gov}, again select
the ``Nuclear Information Service (NIS)", ``Data area"
and ``US ENDF/B Libraries" links, but now
scroll down to ``Energy Balance of ENDF/B-VII.1,'' and click on
``summary.''  This page lists the materials with their problems
and provides links to the plots.

Another test of the processing code comes from running
criticality benchmarks such as those defined in the ICSBEP
Handbook\cite{ICSBEP} using the processed data
libraries.  If results are consistent, it gives some confidence
that the libraries don't contain severe errors.  Table~\ref{crits1}
shows some benchmark results using ENDF/B-VII.0 data processed
with NJOY and used by MCNP5.  The results are calculated
k$_{\rm eff}$ values divided by the evaluated model
k$_{\rm eff}$ value (C/E).

\begin{table}[thbp]
\caption[Criticality benchmark results from MCNP5 with NJOY processed \newline
 ENDF/B-VII.0 data]{Criticality benchmark results from MCNP5 using NJOY
 processed ENDF/B-VII.0 Cross Sections}
\setlength{\extrarowheight}{1pt}
\begin{center}
\begin{tabular}{lll}
Critical & k$_{\rm eff}$ C/E & Description \\ \hline
HMF001 Godiva & 0.99986 & sphere of $^{235}$U \\
PMF001 Jezebel & 1.00006 & sphere of $^{239}$Pu \\
UMF001 Jezebel-23 & 0.99970 & sphere of $^{233}$U \\
HMF028 Flattop-25 & 1.00277 & $^{235}$U core, $^{\rm nat}$U reflector \\
PMF006 Flattop-Pu & 1.00105 & $^{239}$Pu core, $^{\rm nat}$U reflector \\
UMF006 Flattop-23 & 0.99937 & $^{233}$U core, $^{\rm nat}$U reflector \\
IMF007s Bigten & 0.99995 & 10\% enr. U core, $^{nat}$U refl. \\
HST042-02 & 1.00028 & HEU in nitrate solution \\
HST042-03 & 1.00065 & HEU in nitrate solution \\
HST042-05 & 1.00019 & HEU in nitrate solution \\
LCT006-01 & 0.99995 & low enruiched UO$_2$ lattice \\
LCT006-02 & 1.00052 & low enriched UO$_2$ lattice \\
LCT006-03 & 1.00036 & low enriched UO$_2$ lattice \\ \hline
\end{tabular}
\end{center}
\label{crits1}
\end{table}

Similarly, we can calculate criticality benchmarks using
multigroup libraries with a discrete-ordinates transport
code.  The Los Alamos PARTISN code was used with data
prepared using TRANSX.  Table~\ref{mg1} shows results for a
selection of fast critical assemblies using three different
group structures.  Comparing these results with the previous
table shows that multigroup methods can compare favorably
with Monte Carlo when using high-accuracy libraries.
IMF007h is a 2-D cyclindrical homogenized model for
Big-10.  Table~\ref{mg2} shows results for two of the thermal
solution criticals from Table~\ref{crits1}.  Once again, the
results are satisfactory.  These kinds of tests provide some
confidence that the multigroup processing in NJOY is working
properly and is consistent with the Monte Carlo processing.

\begin{table}[tp]
\caption[``FAST" criticality benchmark results from TRANSX/PARTISN
 with NJOY processed ENDF/B-VII.0 data]{Benchmark Results for Fast
 Criticals Using ENDF/B-VII.0 Mulitgroup Libraries and TRANSX/PARTISN}
\setlength{\extrarowheight}{1pt}
\begin{center}
\begin{tabular}{lllll}
Critical & k$_{\rm eff}$ C/E \\
                  & MCNP  &   30 gp   & 80 gp   &  187 gp \\ \hline
HMF001 Godiva     & 0.99986 & 1.00234 & 1.00091 & 1.00025 \\
PMF001 Jezebel    & 1.00006 & 1.00060 & 1.00047 & 1.00036 \\
UMF001 Jezebel-23 & 0.99970 & 1.00115 & 1.00024 & 0.99992 \\
HMF028 Flattop-25 & 1.00277 & 1.00349 & 1.00343 & 1.00295 \\
PMF006 Flattop-Pu & 1.00105 & 1.00126 & 1.00147 & 1.00134 \\
UMF006 Flattop-23 & 0.99937 & 1.00126 & 1.00010 & 0.99971 \\
IMF007h Bigten    & 1.00005 &         & 1.00022 & 1.00001 \\ \hline
\end{tabular}
\end{center}
\label{mg1}
\end{table}

\begin{table}[bp]
\caption[``THERMAL" criticality benchmark results from TRANSX/PARTISN
 and NJOY processed ENDF/B-VII.0 data]{Benchmark Results for Thermal
 Criticals Using ENDF/B-VII.0 Mulitgroup Libraries and TRANSX/PARTISN.}
\begin{center}
\begin{tabular}{llll}
Critical & k$_{\rm eff}$ C/E \\
         &  MCNP   & 69 gp   &   187 gp  \\ \hline
HST042-03 & 1.00065 & 0.99931 & 0.99958 \\
HST042-05 & 1.00019 & 0.99946 & 0.99935 \\ \hline
\end{tabular}
\end{center}
\label{mg2}
\end{table}

As a further verification of NJOY processing, it is useful to do
intercomparisons with other processing systems.  Over the years,
we have periodically compared NJOY results from
\hyperlink{sRECONRhy}{RECONR} and \hyperlink{sBROADRhy}{BROADR}
with the PREPRO system of D. E. Cullen\cite{PREPRO}, especially
when new features have been added to the ENDF system.  We recently
did a number of comparisons with ORNL results to verify the new
capabilities for resonance reconstruction and covariance processing
imported from the ORNL SAMMY code.  In some cases, these
comparisons have been extended to multiple laboratories.  An
example of that is a recent comparison of unresolved resonance
range processing\cite{sublet}.  This study suggests that more work
is needed in the unresolved resonance range.

The comparisons described above were on cross sections.  It is also
useful to do multilab comparisons on integral results.  As an example,
a LANL-LLNL comparison\cite{fast} shows that we can get good
agreement for the fast criticals Godiva, Jezebel, and Jezebel-23.
Another recent multilab comparison involving LANL, LLNL, ANL, and
the CEA demonstrated reasonable results for the softer spectrum of
the Big-10 critical assembly\cite{big10}.

\cleardoublepage

